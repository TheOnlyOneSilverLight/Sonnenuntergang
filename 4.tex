\chapter{Jenseits des Tals}

Das Feuer züngelte empor, leckte an dem feuchten Holz. Es schien, als wäre es verärgert, kein 
besseres Brennmaterial bekommen zu haben und so niedrig sein zu müssen. Dafür spuckte es umso mehr 
Qualm in die Luft. Semái stocherte gedankenverloren darin herum und lauschte dem stetigen Knacken, 
wenn die Luft aus dem Holz entwich. Sie hatten es von einer windschiefen Hütte geklaut, die 
keineswegs verlassen aussah. Irgendeine arme Jägersfamilie, die sich etwas abseits der Siedlung 
eine Unterkunft zusammen gezimmert hatte. Und, wie alle anderen Bewohner dieses Tals, sich bereits 
für das Herbstfest auf dem Dorfplatz eingefunden.
Sie waren nun drei Tage unterwegs und folgten dem steinigen Weg hinauf zum Pass, der sie aus dem 
Tal bringen würde.\\
Semái lächelte grimmig und schüttelte kaum merklich den Kopf über den spielerischen Streit der 
beiden Frauen. Sie warfen sich sarkastische Beleidigungen an den Kopf und das Ziel war wohl, am 
originellsten zu sein. Eigentlich sah kein Gespräch zwischen den beiden anders aus als dieses. Nur 
jetzt kicherte Revo an bestimmten Stellen amüsiert. Semái war sich nicht sicher wie ernst diese 
Geplänker waren. Manchmal machte er auch mit, aber heute hatte er keine Muse dazu. Der Chanahe legte 
den Kopf in den Nacken und versuchte den Weg des Qualms zu verfolgen. Er verschmolz mit der 
Nachtluft und dämpfte das Licht einiger Sterne. Semái fröstelte und zog die Schultern hoch.\\
\textit{Ein wärmeres Feuer wäre schon wünschenswert...}\\
``Hey'', rief er plötzlich überrascht aus und stieß Marai von sich. Der Hund war aus dem Nichts von 
seinen abendlichen Spaziergängen aufgetaucht. Man konnte nur vermuten wo sie sich herumgetrieben 
hatte, aber offenbar war sie dermaßen durch das Unterholz gekrochen, dass ihr Fell nun von 
Schmutz und Nässe troff. Und dementsprechend stank. Semái seufzte, als die Hündin ihn gekränkt und 
bekümmert von seiner Grobheit ansah. ``Ist ja schon gut'', murmelte er und stupste ihr mit dem 
Zeigefinger auf die schwarze Nase: ``Du Hund!''\\
``Woah... das ist die schlimmste Beleidigung, die je ein Mensch auf Erden zu einem Hund gesagt hat, 
du Mensch!'', spottete Inu und warf Marai die Überreste des Abendessens zu.\\
``Warum bist du heute so still?'', fragte Inu: ``Also, noch stiller als sonst?''\\
Semái suchte nach den richtigen Worten, fand sie jedoch nicht. Er würde Inu kränken, wenn er ihre 
Frage ehrlich beantworten würde. Dies war die letzte Nacht, die sie in der Sicherheit des Tals 
verbringen würden. Hier, wo die schlimmsten Kreaturen zu groß geratene Feldratten oder wilde 
Raubkatzen waren. Er war damals noch ein Junge gewesen, aber er hatte die Kadaver der Wesen 
gesehen, die die Chanahe-Vorhut freigeräumt hatte. Er wusste noch genau, wie sich die Nächte 
angefühlt hatten. Die Kinder waren die Einzigen, die schliefen. Sein Vater war damals nicht einmal 
dazu bereit, sich auf dem Boden nieder zu lassen. Stetig lief er um da Feuer und starrte in die 
Nacht. Wohlwissend, dass in der Dunkelheit sie die Opfer waren und nicht die Jäger. Inu hatte 
keinen Schimmer. Ihre Welt existierte bisher nur bis zu dieser Bergkuppe. Er spürte Sarjus Blick 
auf sich ruhen. Eine stumme Aufforderung. Aber er antwortete bloß: ``Nichts. Ich habe nur 
gebetet.''\\
Inu schnaufte spöttisch und verschlang den letzten Bissen ihres Mahls.\\
``Seid wann bist du so gläubig?'', spottete sie.\\
Semái sah sie ehrlich überrascht an und runzelte die Stirn. ``Wieso Glaube? Kanto und Anta haben 
diese Welt erschaffen. Willst du sagen, dass du anderer Meinung bist?''\\
Inu biss sich auf die Lippen. ``Erschaffen ja, aber sie interessieren sich nicht für uns. Wir sind 
ebenso wichtig wie der Dreck unter unseren Füßen!''\\
Sarju räusperte sich. ``Nun ja... sie sind Götter. Man kann es ihnen nicht übel nehmen, dass sie 
nicht regelmäßig zum Abendessen kommen und fragen, ob sie uns einen Gefallen tun können.''\\
``Aber man kann ja hoffen'', murmelte Semái und sein Blick wanderte zum Himmel, suchte Antas 
Mondgestalt.\\
``Sie sind trotzdem nicht die einzigen Götter. Und auch nicht die Mächtigsten'', verkündete Inu und 
lächelte.\\
Semái musste sich ein Grinsen verkneifen und wandte den Blick ab um die verräterische Bewegung 
seiner Lippen vor ihr zu verbergen. Sarju machte sich nicht diese Mühe und lachte kurz laut auf, 
ehe sie sich weiter damit beschäftigte, Muster mit einem Stock in die Walderde zu ritzen.\\
Nur Revo beugte sich zu Inu vor und fragte nach. ``Was meinst du damit?''\\
``Kanto und Anta haben ihre drei Welten erschaffen. Aber es gibt noch viel mehr Götter, außerhalb 
und innerhalb'', erklärte Inu und warf Sarju einen bösen Blick zu: ``Götter, die mehr Macht 
besitzen.''\\
``Und wieso sollten die Schöpfer unserer Welt das zulassen? Ich würde keinen Rivalen in meinem Haus 
dulden'', warf Sarju ein ohne den Blick zu heben.\\
``Weil sie blind sind! Weil sie sich von uns abgewandt haben und sich nicht mehr für uns 
interessieren! Warum sollten wir also noch weiter zwei Göttern hinterher schauen, die uns verlassen 
haben?''\\
``Sie sind die Schöpfer. Warum sollten sie ihre Welt verlassen?'', murmelte 
Semái, wobei er nicht daran glaubte, dass die Schöpfer fort waren. Er hatte ihre Präsenz schon 
einige Male - für Augenblicke, Herzschläge, Atemzüge - deutlich gespürt.\\
``Aber sie haben dir die Gabe verwehrt'', warf Revo ein: ``Zeigt das nicht, dass sie dich für 
wertlos halten? Du bist auch nicht mehr als ein Lisoe!''\\
Semái runzelte die Stirn. ``Ich bezeichne Lisoe nicht als wertlos!''\\
``Aber in den Augen deiner Götter ist es doch so'', spottete Inu und nickte Revo zustimmend zu: 
``Sie haben zwei Arten von Menschen geschaffen. Und die Chanahe halten sich für besser. Begnadet. 
Dabei ist eure Macht so zerbrechlich. Ein Windhauch reicht und die Chanahe sind hilflose kleine 
Kinder.''\\
Semái sah sie überrascht an und versuchte die richtigen Worte für eine Antwort zu finden. Revo kam 
ihm zuvor: ``Ist das dein Ernst? Gibt es eine Möglichkeit, die Gaben zu zerstören?''\\
Bevor Inu etwas erwidern konnte, unterbrach Sarju das Gespräch mit einem lauten Stöhnen. 
``Ernsthaft? Hätte ich gewusst, dass ihr so drauf seid, hätte ich euch niemals angesprochen.''\\
``Sind wir auch nicht'', sagte Semái bestimmt und starrte Inu an. Er hatte sich dagegen 
entschieden, sie hier vor den Zwillingen zu fragen, was plötzlich passiert war. Natürlich hatte sie 
die Chanahe nie leiden können. Aber nie hatte sie Kanto und Anta verleumdet.\\
``Gute Nacht'', beschloss Sarju, ließ sich rücklings aus dem Sitz hinfallen und warf ihren Umhang 
über ihre Augen.\\
Inu biss sich mittlerweile unruhig auf die Unterlippe und schien ihre Worte zu bereuen. Ein 
schnelles, kurzes Handzeichen war die Antwort auf Semáis prüfenden Blick.\\
\textit{Können wir reden?}, bedeutete es und üblicherweise hatten sie es verwendet, um 
herauszufinden ob der andere Zeit hatte sich fortzuschleichen.\\
Semái wandte den Blick ab und legte sich, mit dem Rücken zum Feuer und somit zu Inu, ebenfalls zum 
schlafen hin. 

Inu erhob sich und verließ den Schein des Feuers. Sie taumelte an Bäumen vorbei, spürte die raue 
Rinde unter ihren Fingern und den stechenden Schmerz in ihrem Kopf. Schließlich sank sie nieder und 
presste ihre Handballen gegen die Schläfen. 
``Saraé!'', keuchte sie. \textit{Du musst damit aufhören!}\\
\textbf{Ich habe noch nicht einmal begonnen.}\\
\textit{Wieso lässt du mich solche Dinge sagen?}, verlangte Inu zu erfahren: \textit{Semái... sein 
Blick...}\\
\textbf{Keine Sorge. Er wird nicht gehen. Es gibt für keinen von euch ein zurück, deshalb seit ihr 
doch hier. Jeder einzelne sucht ein Ziel, für das es sich zu reisen, leben, kämpfen lohnt. Und 
deine Aufgabe, Inu, ist es, ihnen dieses Ziel zu geben. Vergiss das niemals! Das ist der einzige 
Grund, wieso du noch einen eigenen Geist besitzt.}\\
Der Schmerz in ihrem Kopf verklang und Inu hielt den Atem an. \textit{Bin ich nur dein Werkzeug?}\\
Leises Lachen. \textbf{Ach nein, mein Kind. Nicht nur.}\\
\textit{Und was ist dieses Ziel?}\\
Inu fasste an das Amulett und spürte, wie das Material sich erwärmte, während Saraés Stimme sie 
fort trug. Der Anblick des Waldes um sie herum verblasste und offenbare weite Wiesen, welche sich 
sanft im Wind wogen. Das Nachmittagslicht legte einen goldenen Schimmer auf die Welt. Die Stille 
wurde je unterbrochen von einem Kind, welches lachend und tanzend seine Kreise zog. Ein Mädchen. 
Ihr braunes Haar wirbelte durch die Luft, während ihre Füße sie hüpfend durch das Grasmeer trugen. 
Ihr wollenes, buntes Kleid bauschte sich mit dem Wind.
\textbf{Du wünscht dir nichts mehr, als Familie, die dich liebt}, wisperte Saraé: \textbf{Deine 
Tochter kann nur glücklich aufwachsen, wenn diese Welt ihren Begriff von Freiheit neu definiert. 
Ohne die Monster der Magie, gibt es keinen Grund, warum Begnadete existieren sollten. Alle Menschen 
wären gleich viel wert. Wenn wir gesiegt haben, kann ich dir das Leben, was du dir wünscht, 
bieten.}\\
Das Mädchen stolperte und fiel. Ihr Gesicht verzog sich vor Schmerz, Tränen schimmerten. Da 
umfassten sie zärtliche Arme, hoben sie hoch und ihr Gesicht verschwand an der Schulter der Frau.\\
\textit{Das bin ich...}\\
Die Gestalt eines Mannes erschien aus dem Nichts und während das wogende Gras ihre Beine umspielte, 
zerzauste er dem Kind das Haar. Inus Herz wurde schwer. Auch wenn sie den Mann nur von hinten sah, 
wusste sie, dass es Semái war. Verwirrt schlug sie die Augen auf und starrte in die Dunkelheit des 
Waldes. Ganz langsam nur verklang das Gefühl des Glücks, welches Saraés Bilder in ihr ausgelöst 
haben.\\
Als sie zum Lagerplatz zurückkehrte, warf sie einen langen Blick auf Semái. Er lag immer noch 
abgewandt vom Feuer, kehrte ihr den Rücken zu. Doch sie wusste, dass er noch wach war und auf ihre 
näherkommenden Schritte lauschte. 

Inu erwachte ruckartig und befand sich schon in einer sitzenden Position, ehe sie erstarrte auf zu 
dem Mann auf sah. Der Chanahe stand über Semái, der mit offenen Augen ausgestreckt auf dem Boden 
lag. Die Klinge seiner Sturmsense bedrohlich nahe seiner Halsschlagader.  Revo schlief noch tief 
und fest, während Sarju sich auf ihr Knie aufgestützt erhoben hatte. Ihr Messer lag locker in der 
Hand, während sie ihre Lippen fest zusammen presste und wartete, was gleich geschehen würde.\\
``Guten Morgen, Meroka'', brach Semái schließlich die Stille und tippte gegen die Klinge. 
``Familienerbstück?''\\
``Natürlich. Gut geschlafen?''\\
Semái runzelte die Stirn und schob die Waffe am Schaft beiseite, ehe er sich aufrichtete. ``Ich 
habe es versucht. Ist die letzte Nacht innerhalb des Tals.''\\
Meroka beachtete die Anderen nicht, sondern streckte Semái eine Hand hin um ihm aufzuhelfen. ``Also 
ist es wahr. Der Rat hat dich verstoßen.''\\
``Vermutlich, ich bin gegangen, bevor sie mich fort jagen konnten.''\\
Meroka kratzte sich am Kinn. ``Dann solltest du dich mit dem gehen beeilen. Heutzutage sind 
Namenlose eine größere Schande als Lisoe.''\\
Semái ersparte sich eine Erwiderung darauf und fragte: ``Was möchtest du loswerden, Meroka?''\\
Der Chanahe zuckte mit den Schultern. ``Ich war da draußen, in den letzten Wochen.''\\
``Also eine Warnung?'', unterbrach er ihn: ``Nun... danke, aber da uns keine andere Wahl bleibt, 
ist sie recht nutzlos.''\\
``Chanahe sein, bedeutet, ein Krieger zu sein. Mit oder ohne Gabe, ein Krieger bist du'', sagte 
Meroka.\\
``Ich sehe mich eher als Poeten'', spottete Semái.\\
Der Chanahe blieb ernst. ``Ich wollte dir nur sagen, dass ich es schade finden würde, wenn wir 
einander nicht wiedersehen.''\\


