
\chapter{Veränderung}


Die Welt hatte sich verändert. Sie war schmutziger, grausamer und dunkler geworden. Das Leben 
trostloser, kürzer und gefährlicher. Irgendwann, Inu konnte nicht einmal den genauen Zeitpunkt 
benennen, war die kindliche Naivität verflogen und hatte ihr die Härte der Realität offenbart. 
Während ihre Schritte den Staub auf der Straße aufwirbelten, glitt ihr Blick über die schiefen 
Hütten aus zusammengeklaubten Unrat. Fünfzehn Jahre war es nun her, dass die Chanahe in das Tal 
kamen. Wie überall waren die Bauern bald zu Sklaven geworden. Die Ernte wurde den Herren überstellt, 
das Vieh, die Arbeitskraft. Es war den Lisoe verboten eigenes Land zu besitzen. Es gab immer noch 
keine fairen Gesetze und Ordnung und durch das enge Zusammenleben in den schäbigen Viertel rund um 
die prächtige Stadt der Chanahe hatten sich die Probleme vervielfältigt. Der Frust und die Angst 
beherrschten das Leben der Menschen. Selbst ein Blick, den die Chanahe als respektlos erachteten, 
könnte einem Lisoe zum Verhängnis werden. Doch am Schlimmsten fand Inu die Ratten. Lisoe die in der 
Stadt wohnten und sich fast schon als Verbündete der Herren fühlten. Sie gingen Handwerke nach, die 
für die Chanahe zu unwürdig waren, während Inu und der Rest die Dreckarbeit erledigte. Die Ratten 
knabberten an den Lebensmitteln, welchen den Lisoe geraubt wurden und lebten ohne jegliche 
Berechtigung dieses Leben der Chanahe.\\
\textbf{Die Zeit der Begnadeten ist bald vorbei}, wisperte Saráes Stimme.\\
Ihre Gegenwart war mit den Jahren vertraut geworden. Inu konnte sich ein Leben ohne ihr nicht mehr 
vorstellen. Es war, als wäre der Geist im Amulett mit ihrem eigenen verschmolzen. Die junge Frau 
schüttelte lediglich den Kopf. Seit Jahren predigte Saraé von dem Ende der falschen Götter, der 
gierigen Drachen die ihre Gaben ungerecht und eigennützig verteilten. Aber es war alles nur 
schlimmer geworden. \\
\textbf{Wir sollten gehen}, drängte sie.\\
Auch darauf ging Inu nicht ein. Auch diese Worte hörte sie seit Monaten. Und die nächste Frage 
gehörte unweigerlich zu dem Monolog der Göttin. \textbf{Was hält dich hier noch?}\\
Dabei wusste Saraé es ganz genau und Inu beschleunigte schweigend ihren Lauf. Aber sie konnte nicht 
verhindern, dass in ihren Gedanken das Bild eines jungen Mannes auftauchte. Ein 
drahtiger Mann, eher überlegend als laut, klug und ruhig. So völlig anders als der Rest der Chanahe. 
Trotz den Unterschieden hatten sie ihre Kindheit miteinander verbracht und er war der einzige Grund, 
wieso sie nicht allen Chanahe den Tod wünschte. Das tat Saraé an ihrer Stelle.\\
``Sei still, ich geh hin und das weißt du'', murrte sie leise.\\
Das Leben der Chanahe war auch nicht erstrebenswert für Inu. Es ging immer um Macht und Eigentum. 
Um Ehre und Sieg. Und diese Dinge stellten sie mit Kampf gleich. Jeder Chanahe musste in die Arena 
um seiner Familie Ehre zu bringen. Für manche war es ein Sport, für Andere bitterer ernst. Semái 
hatte sich Jahre vor diesen Kämpfen drücken können. Seine Schwester, der Inbegriff einer 
kriegerischen Chanahe, hatte diese Familienpflicht übernommen. Eine Zeitlang duldeten seine Eltern 
das, da er ihrer Meinung nach nicht gut genug im kämpfen war und sie ihn weiter versuchte zu 
trainieren. Dabei glaubte Inu gar nicht, dass er schlecht war. Nur zu nett. Manchmal konnte sie nur 
die Augen verdrehen, wenn er sich zurückhielt. Selbst sie war skrupelloser als er. \\
Inu erreichte den Platz, machte sich aber gar nicht erst die Mühe, sich zu den Menschen zu 
gesellen. Die besten Plätze hatten natürlich die Chanahe, die Lisoe drängelten sich in etlicher 
Entfernung zusammen und stellten sich auf die Zehnspitzen. Inu wählte den Weg der meisten, die aus 
ihrem Viertel stammten, und erkletterte eines der steinernen, etwas besseren Häuser am Rande des 
Platzes. Als sie sich setzte und die Füße baumeln ließ, nickte sie den restlichen Leuten auf dem 
Dach grüßend zu und streckte dann den Kopf, um den Platz besser sehen zu können. \\
Inu und Semái sahen sich kaum noch in den letzten Jahren. Anfangs ließen seine Eltern ihn nicht 
mehr aus den Haus und er musste trainieren, später wurde es ein Frevel, mit Lisoe zu sprechen. 
Lange Zeit tauschten sie einander nur mit kleinen, unauffälligen Zeichen aus, bis sie wieder 
Gelegenheit fanden, sich alle paar Tage im Wald zu treffen. Sie wusste also genau, wie es ihm gerade 
ging und welche Angst er vor dem Kampf hatte. Das er verlieren würde, war für Semái schon klar. Die 
Angst richtete sich eher auf seine Familie und deren Reaktion, wenn er sie beschämen würde. \\
Inu wurde von der allgemeinen Aufregung mitgerissen, obwohl sie diesen wöchentlichen 
Veranstaltungen eigentlich widerstrebend beiwohnte. Würde es nicht um Semái gehen, hätte sie einen 
großen Bogen um den Platz gemacht. Die Chanahe wollten mit diesen Kämpfen den Lisoe ihre 
Fähigkeiten zeigen. Zeigen, zu was sie im Stande waren. Es sollte Inu Angst einjagen. \\
\textbf{Sie werden schwächer}, verkündete Saraé wispernd.\\
Im Gegensatz zu Inu liebte die Göttin diese Kämpfe. Sie beobachtete mit einer Aufregung die Chanahe 
wie sonst nichts in Inus Alltag. Viel zu oft schon hatte Saraé sie dazu gedrängt, sich die Kämpfe 
anzusehen.\
\textit{Semái wird versagen. Er wird zu Brei geprügelt werden}\\
\textbf{Aber er ist frei.} Sie lachte hämisch. \textbf{Das ist es doch, was den Begnadeten so am 
Herzen liegt. Die Freiheit, die Kanto ihnen gab. Die Freiheit, alle anderen Menschen zu 
versklaven. Ist das nicht ein Inbegriff der Ordnung? Wer steht oben, wert steht unten? Es ist 
alles geregelt, kein Lisoe kann aus seiner Rolle weichen. Die Freiheit verschwindet und die 
Ordnung zieht mit ihren Gesetzen ein. Nur ergibt sich niemals Frieden, aus dem Wunsch nach 
Freiheit und Ordnung. Nein, das Chaos blüht. Es ist wie das Auge eines Sturms. Im Inneren, ein 
winziger Punkt, herrscht die Ordnung. Alle anderen - die Lisoe - werden vom Chaos verschlungen.}\\
``Sei still jetzt'', murmelte die junge Frau. \\

(Semáis Sicht)\\
``Bist du bereit?'', fragte die Frau. \\
Ihr langes schwarzes Haar umrahmte ihr blasses, hübsches Gesicht. Doch langsam zeichnete die Zeit 
ihre Spur in die Haut. Semái blickte in die dunklen Augen seiner Mutter und wusste nicht, was er 
auf 
diese Frage hin antworten sollte. \\
\textit{Nein}, dachte er nur, aber das wusste seine Mutter auch so.\\
Keiner in der Familie glaubte daran, dass Semái auch nur einen Hauch einer Chance hatte zu 
gewinnen. 
Das Ziel war eher, dass er möglichst lange durch hielt. Er hatte noch lebhaft die Diskussionen 
seines Vater in den Ohren, wie er alles versucht hatte diesen Kampf zu verhindern. Aber seine 
Kollegen im Senat wollten ihre Familie blamieren und sie konnten nichts dagegen tun. Sein Vater 
hatte ihm geschworen, dass sie ihn nicht verjagen würden. Er hatte wohl einen Pakt mit den anderen 
Senatsangehörigen geschlossen. Allein auf diese Vereinbarung hinaus nahm er überhaupt an dem Kampf 
teil. Auch wenn sein Vater ein stolzer Mann war, er hätte nicht zugelassen, dass man seinen 
ältesten Sohn verbannte. \\
\textit{Hauptsache sie blamieren die Familie.}\\
Semái seufzte und wandte sich von seiner Mutter ab. Der Wunsch war groß, einfach zu gehen. Aber was 
sollte danach geschehen? Er schüttelte frustriert den Kopf und zog das Stoffband, welches sein 
Handgelenk stützten sollte, fester. \\
``Es geht in einigen Minuten los. Sei dann bereit'', sagte sie schließlich und verließ den 
Vorbereitungsraum. \\
``Oh'', entfuhr es ihm, als sie außer Hörweite war: ``Ich bin erledigt.'' \\
Der junge Mann dachte an die ganzen Lektionen und Predigten die sein Vater ihm eingebläut hatte. 
Die 
Technick des Kampfes gelang ihm gut, aber ihm fehlte einfach etwas um die Perfektion zu erreichen. 
Oder auch nur ansatzweise so gut zu sein wie alle anderen Chanahe. Seine Schwester Mibell war 
miserabel was die Technik anging. Sie war zu unkontrolliert. Trotzdem beendete sie jeden Kampf in 
kürzester Zeit in dem er nach Luft ringend auf dem Boden lag und nur noch versuchte seinen Kopf mit 
den Armen vor weiteren Tritten zu schützen. Mibell war schlicht schneller, beweglicher und sicherer 
im Stand. \\
``Vielleicht bin ich ja wirklich nur zu nett'', murmelte er und grinste. Es wäre so schön, wenn das 
die Antwort auf seine Probleme wäre. Nett sein kann man immerhin leichter aufhören.\\
\textit{Kanto... Anta... wieso habt ihr mich mit der Gabe ausgelassen?}, zweifelte er 
und konnte wieder nur den Kopf schütteln. Diesmal würde er nicht nett sein. Sein Gegner war 
immerhin nicht seine Schwester sondern ein Kind einer ehrbaren Familie, das es genießen würde, 
Semáis Namen in den Dreck zu ziehen. Gleich nachdem er sein Gesicht in den Dreck gedrückt hat. \\
``Na, großer Bruder? Aufgeregt?''\\
Er hob den Kopf und nickte Mibell als Antwort nur zu. Ihr dunkles Haar hatte sie streng 
zurück gebunden, damit es auch ja nicht störte. Sie spielte sogar mit den Gedanken, sie komplett 
abzuschneiden um Gegnern weniger Möglichkeiten zu geben sie daran zu packen. Semái hoffte, dass sie 
das nicht tun würde. Er mochte ihr Haar und fand, es war das Schönste an ihrem Äußeren. \\
``Sie ist übrigens da. Die Lisoe die du so magst. Ich habe sie auf einem Dach gesehen.''\\
Er richtete sich gerade auf und lächelte flüchtig. Mibell musterte ihn kritisch, sagte aber nichts 
negatives. Sie duldete diese Freundschaft, hatte sogar einige Male als Vermittlerin für ihren 
Bruder 
und die Lisoe gedient, wenn er wieder nicht aus seinem Zimmer durfte.\\
``Du willst sie aber nicht, oder? Also… du liebst sie nicht?'', fragte sie leise.\\
Der Tonfall in ihrer Stimme vermittelte, dass diese Frage ihr wohl schon eine Weile auf dem Herzen 
gelegen hatte. Semái dachte kurz über die Frage nach und schüttelte dann den Kopf. ``Nein.''\\
``Und sie dich?''\\
Er lachte kurz auf. ``Nein! Wir sind wie… Geschwister.''\\
Mibell verschränkte die Arme vor der Brust. ``Ist ja nicht so, als ob du schon eine tolle Schwester 
hättest!''\\
Grinsend legte er ihr einen Arm um die Schulter. ``Sie verprügelt mich immerhin nicht so oft. Nur 
manchmal!''\\
Mibell legte ihren Kopf an seine Schulter. ``Ich hasse es… wenn du wenigstens mal zurück schlagen 
würdest, würde ich mich nicht so schlecht fühlen.''\\
``Ich gebe mein Bestes'', scherzte er. \\
``Pass auf dich auf'', flüsterte Mibell: ``Ich habe Angst um dich.''\\
Tief holte er Luft und löste sich dann von ihr. ``Weißt du? Genau das will ein Mann hören. Dass 
seine kleine Schwester ihn beschützen will.''\\
``Das habe ich so nicht gesagt.''\\
Er sah sie nur ruhig an, bis sie den Blick senkte und ihre Absicht so zugab. \\
``Versuch sie von den Beinen zu holen.''\\
``Als ob das bei einem Chanahe so leicht wäre.''\\
``Du musst nur diesen Kampf durchstehen. Danach musst du nie wieder in die Arena'', sagte Mibell.\\
``Ach ja? Vor ein paar Wochen hieß es auch so, dass ich das nicht machen muss, weil du für die 
Familie antrittst'', antwortete er finster und erhob sich. Er nickte nur und schritt dann zum 
Eingang in die Arena hinaus. \\


Semái wusste von der Wirkung seines Auftretens. Er wirkte wie stets ruhig und gefasst, während der 
Blick seinen dunklen Augen die Arena studierten. Vermutlich wusste nur eine einzige Person die ihn 
hier stehen sah, dass er all seine Willenskraft aufbieten musste um nicht zu zittern. Dass er den 
schnellen Puls seines Herzens spürte und sein Magen sich anfühlte, als hätte ihn jemand 
verknotet.\\
Ohne das es eine bewusste Handlung war, huschte sein Blick zu den Dächern empor und fand diese 
bestimmte Person sofort. Inu ragte für ihn aus der Menge empor. Äußerlich war sie wie jede andere 
der Lisoe. Unscheinbar, braun und schmutzig. Nur die Perlen in ihrem Haar waren die einzigen 
Farbkleckse zwischen ihrem braunen Haar, der sonnengebräunten Haut und der ledernen Kleidung. Er 
entsann sich noch gut an den Tag, als er ihr die bunten Holzperlen schenkte. Er hatte sie Mibell 
geklaut, aber für sie waren es nur Schmuck für ihre Puppen gewesen, die sie schon damals kaum 
angesehen hatte. Sie hatte vermutlich nicht einmal bemerkt, dass sie verschwunden waren. Inu hatte 
die Geste fast schon gleichgültig akzeptiert, aber Semái kannte sie gut genug um zu wissen, wie 
gerührt sie von diesem Geschenk war. Sonst hätte sie die Perlnen niemals in ihr Haar geflochten.\\
\textit{Etwas kostbares, verborgen im Schlamm}, dachte er und lächelte schief.
Er hob die Hand und das Zeichen, welches seine Finger formten, war nur so kurz und flüchtig, dass 
es kaum jemandem außer Inu aufgefallen sein konnte. Und selbst wenn, hätte keiner die Bedeutung 
verstanden. Es war eines der vielen Zeichen mit denen sie sich verständigten, wenn sie in Arashad 
aneinander vorrüber gingen, ohne dass sie sprechen konnten. Es waren viel zu viele Jahre dabei 
gewesen, in denen Semáis Eltern ihn nicht aus den Augen ließen. Dieses Zeichen eben war ein stummer 
Gruß, der nicht mehr bedeutete als ``Mein Herz sieht dich.''. \\
Semái hatte sich diese Zeichen ausgedacht und sich bei diesem bestimmten von den Aufzeichnungen 
eines längst verstorbenen Poeten inspirieren lassen. Inu wusste vermutlich nicht, dass diese Worte 
einem Gedicht entsprungen sind. Dabei hatte es sich aber nicht um ein Liebesgedicht gehandelt, 
sondern von einem Mann, der in jungen Jahren auf Reisen ging und nie wieder heimkehrte, denn seine 
Heimat wurde zerstört. Doch das Abbild seiner Heimat, seines Ursprungs, sah er jede Nacht in seinen 
Träumen. Und Träume, dass wusste jeder, ruhten im Herzen und nicht im Kopf. Für die beiden Freunde 
war dieser Gruß ein Versprechen und ein Schwur, der inen in den einsamen Zeiten so viel Hoffnung 
gegeben hatte. \\
Die Chanahe in den forderen Reihen begannen zu tuscheln. Semái biss sich auf die Zunge und wandte 
sich seinem Gegner zu. Die Frau war vermutlich in seinem Alter. Großgewachsen für eine Frau, aber 
drahtiger als Mibell. Sie trug die selbe Kampfkleidung wie er, daher beachtete er ihre Ausrüstund 
nicht weiter. Im Endeffekt war es eh egal, Semái sah keine Chance auf Sieg für ihn. Sein Plan war, 
dass alles einfach über sich ergehen zu lassen. Die Frau hatte auffallend helles blondes Haar, 
welches in zerzausten Wellen bis zu den Schulterblättern reichte. Einzelne Strähnen standen wirr 
vom Rest ab. Sie war hübsch, wie es für Chanahe üblich war. Hohe Wangenknochen, gerade Zähne und 
glatte Haut. Aber ihre Augen waren es, die Semái doch genauer hinschauen ließen. Strahlend blau und 
voller Hass. Er kannte sie nicht. Aber trotzdem hatte er einen Gegner erwartet, der schon von 
Beginn hat tryumphierend lachen oder spottend auf ihn blicken würde. Jemanden, der das Kämpfen 
genoss und die Aufmerksamkeit der Zuschauer liebte. Eben jemanden, der gerne in der Arena stand. 
Wie Mibell. Aber die blonde Frau war nicht aus Liebe zur Arena hier. \textit{Sie hat einen Grund.}\\
Sie wirkte zielstrebig und ehrgeizig. Semái runzelte die Stirn; versuchte zu ergründen, woher der 
Hass in ihren Augen kam. Und was das Ziel dieses Hasses war. Seine Erniedrigung? Er konnte sich 
wirklich nicht entsinnen, diese Chanahe schon einmal in Arashad gesehen zu haben.\\
Der Richter betrat das Feld um die Kontrahenten dem Publikum vorzustellen. Er deutete auf die 
blonde Frau und rief mit einer Stimme, die ohne wie schreien klingen zu müssen, das 
Tuscheln der Zuschauer mühelos übertönen konnte: ``Sarju Mahel Alkarsan aus dem Hause Alkarsan. Die 
werte Familie ist erst vor wenigen Wochen zu unserer Gemeinschaft in Arashad dazugestoßen. Sie 
tritt an gegen Semái Jel Malaza aus dem Hause Malaza, Gründerhaus unserer Stadt Arashad und 
Vertreter im Senat.''\\
Der Richter trat zurück und verschwand durch eine Tür um kurz darauf in der ersten Reihe der 
Tribüne wieder aufzutauchen. Er hob beide Hände und die Zuschauer verstummten nun entgültig. Semái 
ging leicht in die Knie und verlagerte sein Gewicht gleichmäßig auf beiden Beinen, während er auf 
das schnelle Ende wartete.\\
``Die Beteiligten haben sich geeinigt, dass der Kampf Waffenlos stattfinden wird'', verkündete der 
Richter und gleich darauf vernahm Semái deutlich die Enttäuschung der Zuschauer. Sie hatten auf 
viel Blut gehofft. Aber da hatte vermutlich Semáis Vater seine Finger im Spiel gehabt. Einer 
wirbelnden Klinge geführt von einem rücksichtslosen Chanahe wäre Semáis Tod. Mit Fäusten und 
Tritten könnte er vielleicht mit einer gnädigen Ohnmacht davon kommen. Der Richter ließ seine Arme 
fallen und im selben Moment erklang das absurd zarte Klingen eines kleinen Glöckchens.\\
Sarju Mahel Alkarsan aus dem Hause Alkarsan federte sich mühelos vom Boden ab. Ihre langen Finger 
ergriffen eine der vielen Stangen die in verschiedenen Höher in der Arena angebracht waren und von 
Oben aussahen als würden sie sich wie ein Netz überlagern. Die Stangen besaßen unterschiedliche 
Dicke und Stabilität. Sie forderten die Geschicklichkeit der Begnadeten heraus, versuchten sie zu 
Leichtsinnigkeit anzustiften. Das Geflecht begann in zwei Meter höhe, sodass ein Kampf am Boden 
trotzdem noch möglich war. Oder vielleicht hatte es eher den Grund, dass ungeschickliche Chanahe 
für ihre Fehltritte mit Verletzungen gestraft werden konnten. Sarju zog sich in einer fließenden 
Bewegung zwischen den Stangen empor, bis sie aufrecht stand. Ihr Stand war sicher und ihr Körper 
ausbalanciert, obwohl es bei der Dicke dieses Stabes vermutlich für einen Begnadeten noch gar nicht 
notwendig gewesen wäre. Semái fluchte leise. Dass was er sah bedeutete, dass die Fremde nicht wie 
Mibell war oder wie viele andere junge Begnadete in Arashad. Diese Chanahe wusste von ihren 
Fähigkeiten und deren Grenzen. Sie überschätzte sich nicht so maßlos wie viele andere Begnadete und 
legte Wert auf die Technik ihres Kampfes.\\
Semái kniff die Lippen zusammen und legte den Kopf in den Nacken um zu der Frau aufzusehen. Sie 
schien auf ihn zu warten und wurde ungeduldig. Ihre blauen Augen waren für Semáis Geschmack zu 
grell, als dass er dieses Körpermerkmal an ihr als Hübsch bezeichnet hätte; und dass sich diese nun 
skeptisch verengten hob das nur noch weiter hervor.\\
``Komm hoch, Malaza!'', rief sie laut aus und machte eine einladende Handbewegung.\\
Semái schwieg und verharrte. Das Tuscheln der Menge schwoll wieder an. Wütende Rufe und belustigtes 
Lachen drang an seine Ohren. Das hier war fast schon eine größere Blamage als wenn die Fremde 
Chanahe sich sofort auf ihn geworfen und ihn nieder gemacht hätte. Stattdessen stand sie immernoch 
in der Höhe und zögerte, überlegte. Plötzlich schoss sie zwei Schritte vor. So schnell und 
unerwartet, dass Semái keinen Schritt machen konnte, ehe sie sich schon hinter ihm hatte zu Boden 
gleiten lassen. Er spürte Sarjus Atem im Nacken und ihre Fäuste pressten sich gegen seinen Kehlkopf 
und seinen Bauch. Der junge Chanahe kam nicht umhinn, während er nach Luft schnappte, diesen Griff 
zu bewundern.\\
``Ist das ihr ernst?'', flüsterte sie in sein Ohr: ``Sie schicken einen Lisoe in die Arena?''\\
``Meine Familie ist alt und mächtig'', murmelte Semái doch etwas gekränkt.\\
``Das ist es ja, was mich so wütend macht. Meine Familie ist reich und groß. Bei uns wird nicht 
lange gezögert. Missgeburten werden entsorgt, noch ehe man ihnen einen Namen gibt. Unbegnadete 
werden verbannt. Das Haus Alkarsan ist groß.''\\
``Das heißt die Mütter deines Hauses schicken ihre eigenen Kinder in den Tod?''\\
Sarjus griff verstärkte sich und nahm Semái die Luft. Seine Hand fuhr reflexartig zu seiner Kehle, 
da lockerte die blonde Frau ihre Faust und gewährte ihm wieder Atemluft. \\
``Nein! Wieso stecken sie dich in die Arena?''\\
``Politik.''\\
``Selbst wenn ich noch was von dir übrig lasse, wäre dein Name unrein. Es gibt für dich keine 
Chance Ehre zu erlangen'', sagte sie.\\
``Als du in die Arena kamst, lag Hass in deinen Augen'', sprach Semái so ruhig wie es nur ging, 
während einem eine Faust gegen die Kehle gedrückt wurde: ``Du bist neu in Arashad. Du kennst meine 
Familie nicht, also kann der Hass nicht für mich sein. Oder doch?''\\
Sarju zögerte, dann stieß sie ihn von sich und sprang einen Satz nach hinten. Semái hob die Arme um 
schneller sein Gleichgewicht wieder zu finden. In den Augen eines Chanahe wirkte sein Taumeln 
umbeholfen, während er doch schneller als ein untrainierter Lisoe wieder sicher stand und sich der 
Gegnerin zuwandte. \\
``Du hast recht'', rief Sarju so laut, dass auch die Zuschauer auf den Tribünen sie hören konnten: 
``Das wäre mal was anderes. Lass uns am Boden bleiben.''\\
\textit{Wieso tut sie das? Dieser erbärmliche Versuch meine Ehre zu retten... }\\
Semái blickte sie an und versuchte ihre Absicht, ihre Seele, zu ergründen. Woher kam diese junge 
Frau? Was trieb sie hier her in diese Arena? Was ist ihr Ziel und warum gab sie ihm eine Chance? 
Kein Chanahe den Semái kannte hätte das für einen Fremden getan. Auch nicht Mibell. Seine Schwester 
gehorchte ihren Eltern, ihrer Pflicht. Sie hielt sich zurück, weil sie Semái liebte. Aber einen 
Fremden? Niemals. Mibell hätte ihn fertig gemacht und auf ihn gespuckt. Semái sah stolz in ihren 
Augen. Oder doch nur Trotz? Und Zorn, welcher den eben noch deutlich spürbaren Hass überlagerte. Er 
biss die Zähne zusammen und seine Antwort war ein stummes nicken. \\
Er wertschätzte diese Chance die sie ihm gab. Nicht um seine Ehre zu verteidigen, sondern weil es 
das erste Mal sein wird, dass er mit jemandem auf einer Ebene kämpft. Falls sie es wirklich schafft 
ihre Gabe dermaßen zurück zu halten.\\
Die beiden Chanahe umkreisten sich. Die Zuschauer und der Richter, die Familien und das Gemurmelt 
war vergessen. Es gab nur noch Sarjus blaue Augen, dass zittern ihrer Muskeln und ihre 
schleichenden Bewegungen. Sie wartete, erlaubte Semái den ersten Angriff. Er tat ihr diesen 
Gefallen. \textit{Sie unterschätzt mich wie jeder hier.}\\
Semái sprang vor. Er hob die geballte Faust und schlug auf ihren Kopf. Die Chanahe war überrascht, 
dass ihr Gegner gleich dermaßen attakieren würde, wich dem Angriff jedoch mühelos aus. Zu schnell 
für einen normalen Menschen. Aber er machte ihr keine Vorwürfe. Das was sie vorhatte wäre, als wenn 
man einem Lisoe sagt, er soll nur noch sein rechtes Bein benutzen. Und in der Sekunde in der sie 
vermutlich darüber nachdachte, dass sie sich nicht fair benahm, traf Semaís Tritt ihre linke 
Kniekehle, die sie ihm in ihrem Ausfallschritt zugewandt hatte. Sarju Mahel Alkarsan ging in die 
Knie, stützte sich mit einem überaschtem Auflachen auf ihr rechtes Knie und blickte zu Semái auf. 
Dieser hatte längst wieder Abstand zu ihr gesucht und blieb in einer wachsamen 
Verteidigungsstellung. Immer noch grinsend schüttelte sie den Kopf. Ihre Haarsträhnen wirbelten 
sanft durch die Luft. Sie stemmte sich auf und klopfte den Staub von ihrer ledernen Hose. ``Das 
habe ich verdient, hm? Wer seinen Kopf hoch trägt sieht nicht, wenn er in Scheiße tritt.''\\
Das Zitat gefiel Semái und entlockte auch ihm ein grimmiges Lächeln. Noch einmal ließ sie ihm nicht 
freiwilligen den Vortritt. Die Schläge und Tritte die sie austauschten gingen fließend in einander 
über. Semái änderte schnell seine Taktik und anstatt auf Abstand zu ihr zu gehen, wich er kaum von 
ihr zurück. Zeitweise rangen sie miteinander, dann folgten wieder Schläge und Tritte. Semái ging 
deutlich öfter nieder in den Staub als die Begnadete, aber sie verdankte ihm eine blutende Lippe 
und das war es wert. \\
Sarju hatte die Oberhand und hielt ihn fest im Griff, als ihre Lippen wieder ganz nah an sein Ohr 
kamen und sie flüsterte: ``Und wie soll es ausgehen?''\\
``Sie werden es uns nicht abkaufen, wenn ich gewinnen sollte'', keuchte Semái angestrengt.\\
Auch das war einfach eine Tatsache, die er hinnehmen musste. Er hatte viel mehr eingesteckt. Sein 
Körper schmerzte, seine Muskeln brannte und selbst das Atmen tat weh.\\
``Wir sollten diesen Tanz wiederholen'', erwiderte die blonde Frau.\\
Semái schaffte es, sich aus ihrem Griff zu finden und schlug in die Magengegend. Doch sie wich mit 
der leichtigkeit eines fallenden Blattes aus. Seine Faust traf sie zwar, aber glitt an ihrer 
Schutzkleidung ab. Ihre Faust dagegen donnerte burtal auf seinen Brustkorb alle Luft entwich aus 
seinen Lungen. Semái klappte zusammen und blinzelte die Tränen weg, während er mühsam nach 
Sauerstoff rang. Die Sekunden verstrichen. Selbst wenn er sich die Mühe gemacht hätte, hätte er es 
dieses Mal wohl nicht rechtzeitig wieder auf die Beine geschafft. Das zarte Klingen der Glocke 
erklang und leutete das Ende dieses Tanzes ein. Sarju Mahel Alkarsan hob die geballte Faust zum 
Himmel und drehte sich im Kreis. Sie schenkte den Zuschauern endlich die Aufmerksamkeit, nach der 
sie die ganze Zeit schon gierten. Jubeln und Pfiffer erklangen, während Semái sich langsam auf die 
Seite rollte und sich bemühte nicht zu kotzen. \\


``Dein Vater hat weggeschaut, als du hingefallen bist.''\\
Semái seufzte nur. ``Bei welchem Mal?''\\
Inu ignorierte den trockenen Humor und strich ihm vorsichtig das Haar aus der Stirn um eine der 
Beulen zu betrachten. ``Beim letzten Mal.''\\
Er ließ den Kopf noch weiter hängen. \\
``Hat das bei diesem heiligen Geschlecht der Begnadeten irgendeine tiefsinnige Bedeutung?''\\
``Schwäche ist unehrenhaft. Es bedeutet nur, dass für ihn die Blamage zu groß war um den Anblick 
meiner Schwäche ertragen zu können. Er hat also allen Menschen in Arashad gezeigt, dass er mich für 
unehrenhaft hält. Er hat damit ausgedrückt, dass meine Schwäche nichts mit ihm und seinem Haus zu 
tun hat...''\\
``Was hat er denn erwartet?'', rief Inu erbost auf: ``Dass du plötzlich doch deine Gabe hast und 
die Blonde in den Dreck wirfst?''\\
Semái zuckte mit den Schultern und verzog das Gesicht vor Schmerzen. ``Vielleicht...''\\
Inu betrachtete eingehend sein Gesicht. Die vielen Schrammen und Kratzer, die Beulen und 
Prellungen.\\
\textbf{Er hat gut gekämpft.}\\
\textit{Nicht gut genug für die Begnadeten.}\\
``Was heißt das jetzt? Was wird jetzt passieren?'', fragte sie.\\
Semái schüttelte wieder den Kopf. Inu schaufte empört und erhob sich. Vor ihnen ruhte der grüne 
Waldsee, der zu einen ihrer vielen gemeinsamen Orte geworden war. Es war hier zu still hier, aber 
Semái konnte Stunden damit verbringen hier einfach nur in das Wasser zu starren. Sie ließ ihm meist 
diesen Spaß, aber heute hatte sie genug. Sein Kummer erfüllte sie mit Zorn. Er saß da und jammerte, 
leckte seine Wunden. Wunden, die völlig sinnlos zugefügt wurden. Nur weil ein paar Idioten 
langeweile hatten. ``Stehst du drauf, dass dich die ganze Stadt als minderwertig betrachtet? 
Gefällt dir das? Selbst deine Eltern würden auf dich spucken, wärst du nicht die Furcht ihrer 
Vereinigung. Meinst du, dein Vater hätte dich als Säugling ersäuft, wenn er gewusst hätte, dass du 
kein Begnadeter sein wirst? Das du Schande und Schwäche über ihn bringen wirst?''\\
Inu hatte ihn in der Arena nicht aus den Augen gelassen. Sie hatte es gesehen. Auch wenn er viel zu 
oft zu Boden ging, hatte ein Einklang zwischen den Kämpfenden bestanden. Sie hatten Spaß gehabt, 
sich an einander zu messen, auch wenn diese blonde Frau unbestreitbar überlegen war. Inu hatte ihn 
noch nie kämpfen sehen und als sie vor wenigen Stunden dort auf dem Dach saß, war sie erschrocken, 
wie gut er war. Für einen Lisoe. \\
\textbf{Was willst du? Das er dich schlägt?}\\
Seine Verärgerung zeigte sich nur in seiner Mimik, ansonsten blieb der junge Mann ruhig am Seeufer 
sitzen und blickte zu ihr auf. Seine Stimme zitterte leicht, als er antwortete: ``Bist du wütend, 
weil ich besser damit klar komme eine Missgeburt zu sein als du? Die Chancen sind gering, dass 
Nachkommen von Chanahe so machtlos wie Lisoe sind. Zufall. Ich habe mich damit abgefunden. Ich bin 
eine Abweichung des Systems. Das passiert. Aber dein Schicksal war es von anfang an, eine Lisoe zu 
sein. Du hattest überhaupt keine Chance. Ich werde bemitleidet, weil ich die Abweichung bin. Du 
wirst verachtet, weil du das vorhersehbare Ergebnis der Gleichung bist. Mein Vater schämt sich für 
mich. Er gibt sich selbst die Schuld dafür, was ich bin. Er gibt sich die Schuld daran, mich in so 
ein Leben gebracht zu haben. Ja, vielleicht hätte er mich umgebracht, wenn er es gewusst hätte. 
Aber nicht aus Eigennutz, sondern aus Liebe. Es gibt genügend Lisoe die ihre Säuglinge umbringen, 
nur um weniger Mägen füllen zu müssen. Das ist der Unterschied zwischen uns, Lisoe. Und noch einer. 
Ich komme damit klar, dass ich keine Gabe besitze. Ich bin zufrieden mit mir, während du immer auf 
der Suche bist. Du kämpfst gegen alles und nichts! Du hasst mein ganzes Volk und verachtest dein 
eigenes!''\\
Wüt erfüllte ihr Herz. Doch stand sie hilflos dort am Ufer, ballte nur ihre Fäuste und suchte nach 
Worten, die ihm ebenso schmerzen würden wie seine. Ihm entging ihre Hilflosigkeit nicht und nun 
erhob er sich doch. Er war nur wenig größer als sie, aber muskulöser, besser ernährt. Er hatte nie 
Hunger gelitten. Stattdessen aber wurde er geschlagen und eingesperrt. Was war besser? Was war 
schlimmer? Was machte wütender und was brachte mehr Kummer?\\
``Scheiße. Hättest du mir nicht einfach eine rein hauen können?'', fauchte sie: ``Weißt du was? Du 
hast doch eine Gabe, Chanahe! Deine Worte verletzen mehr als Schläge es könnten!''\\
``Wieso bist du so unzufrieden?'', fragte er und ignorierte ihre letzten Worte.\\
``Warum bist du es nicht? Was hat das Leben schon zu bieten? Wir sitzen hier fest in dieser 
beschissenen, stinkenden Stadt. Kein Mensch hält uns für wertvoll. Keiner beachtet uns. Du wirst 
nur beachtet, um deine Familie zu beleidigen. Ich? Ich bin die stinkende Lisoe.'' Sie breitete ihre 
Hände aus. ``Das hier! Das ist nicht das ,was ich will. Mein Leben kann nicht einfach hier enden. 
Was soll noch geschehen? Ich heirate, setze eine handvoll Kinder in diese trostlose Welt und sterbe 
dann, nachdem ich sie jahrelang getadelt habe. Und was wartet auf dich hier in Arashad? Eine 
Familie, die sich schämt. Feinde deines Vaters, die jede Gelegenheit nutzen dich zu erniedrigen. 
Und du hörst sie vielleicht nicht, weil du es nicht hören willst, aber es gibt viele Chanahe die 
der Meinung sind, dass Zufälle wie du nicht leben sollten. Willst du, dass ich die nächsten Jahre 
dastehe und zuhöre, wie sie flüstern? Soll ich daneben stehen, wenn dich jemand umbringt oder dein 
Vater dich verstoßt?''\\
Semái lachte spöttisch. ``Das ist es also. Das selbe wie immer. Du willst gehen und du willst das 
ich mitkomme. Wohin willst du gehen?''\\
Immerhin war es das erste Mal, dass er diese Frage stellte. ``Selbst wenn es nirgends besser ist 
als hier, sind wir wenigstens die Chanahe los.''\\
Sie erkannte in seinen Augen, dass dieses Argument wertlos war. Verzweifelt gestikulierte sie und 
suchte nach Worten... bis plötzlich Saraé ihre Zunge übernahm und zum ersten Mal persönlich mit 
einem anderen Menschen sprach. ``Diese Welt ist groß und weit. Sie steckt voller Geschichten und 
Geheimnisse. Voller Magie und Schicksal. Manchmal muss man erst fliehen um eine Heimat zu finden. 
Alles ist ungewiss, nur nicht, dass diese Stadt uns fremd ist. Wir sind hier nicht willkommen. Wenn 
dir Ehre so wichtig ist, Chanahe, dann kehre deinem Haus jetzt den Rücken zu und warte nicht, bis 
der Senat deinen Vater zwingen kann, dich zu verstoßen oder umzubringen. Die Zeiten haben sich 
geändert, nun ist der Moment da, in dem wir unseren Weg ändern müssen. Die Zeit des Aufbruchs ist 
da.''\\
\textit{Du hast die gesagt, dass du Semái dabei haben willst}, dachte Inu überrascht.\\
\textbf{Da hatte ich ihn auch noch nicht kämpfen sehen.}\\
Semái wandte sich ab und sein Blick schweifte über die Lichtung. ``Heimat ist dort, wo man 
willkommen ist'', murmelte er leise.\\
Inu wollte zustimmen, doch Saraé hielt sie zurück. \textbf{Schweig! Alles was gesagt werden musste, 
wurde ausgesprochen. Lass ihn nachdenken.}\\
Die junge Frau gehorchte und setzte sich wieder in das saftige Gras. Während Semái schwieg, drehte 
sie eine ihrer zahlreichen Zöpfe auf und flocht ihn neu. Es vergingen wenige Minuten, bis er 
schließlich wieder etwas sagte. ``Wohin, Inu? Wo ist das Ziel?''\\
``Keine Ahnung. Was willst du denn mal sehen? Das Ende der Welt? Die höchsten Berge? Die tiefsten 
Schluchten?''\\
``Du kennst nur deine Siedlung und Arashad. Du hast keine Ahnung wie der Rest der Welt aussieht. 
Ich weiß immerhin, dass die nächsten STädte der Chanahe Wochen von hier entfernt sind. Und 
dazwischen sind nur Siedlungen wie die deine. Siedlungen, die sich bekriegen. Räuber, Mörder, 
Diebe. All die Dinge, von denen die Chanahe sich abgrenzen wollen. Darum bauen sie die Städte und 
darum halten sie alle Lisoe für erbämrlich und schlec
\textbf{Guter Plan.}\\
\textit{Ich frage mal lieber nicht, ob du den Weg zum Ende der Welt wirklich kennst.}\\
``Dann... wann geht's los?'', fragte Inu voller Tatendrang.\\
Doch kurz wurde sie stutzig und horchte in sich hinein. Noch am Morgen diesen Tages war sie völlig 
gegen eine Reise gewesen. Saraé entging ihr Zögern nicht. Sachlich sprach die Göttin in ihren 
Gedanken: ``Mir gehören deine Gedanken, dein Blut, dein Glaube und deine Seele. Dein Herz weigerte 
sich, aber da Semái nun mitkommt, sind wir uns ja endlich einig.''\\
Inu hatte das Gefühl, dass sie widersprechen sollte, aber es gelang ihr nicht.ht.''\\
``Gleichfalls'', erwiederte Inu sarkastisch.\\
Semái lachte, als fände er seine eigene Entscheidung lächerlich. ``Also gut. Dann lass uns mit dem 
Ende der Welt anfangen, würde ich sagen.''\\
