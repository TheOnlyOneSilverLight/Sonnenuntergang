\chapter{Abschied}


\textit{Abschied bedeutet, dass man etwas aufgibt, was man liebt.}\\
Das Mondlicht schimmerte durch das geöffnete Fenster und erhellte den Raum, der in den letzten zehn 
Jahren seine Zuflucht und viel zu oft auch sein Gefängnis war. Während Semái sich in seinem Zimmer 
umsah, keimte ein Samen der Verbitterung in seinem Herzen. Das hier war bisher sein zuhause 
gewesen, aber nichts zeugte von Liebe. Der Raum war schlicht eingerichtet, mit allem notwendigen 
und auch einige Dinge, die nur für die Annehmlichkeit gedacht war. Aber nichts persönliches. 
Nichts, was seinen Charakter oder seine Träume widerspiegelte. Er saß da und suchte. Suchte nach 
etwas, was sich lohnen würde mitzunehmen. Er packte schließlich nur ein abgegriffenes Buch zu 
seiner Kleidung, der Decke und dem kleinen Päckchen Proviant. Es enthielt die Gedichte des Poeten 
Mirhe Sovatis, der auch das Gedicht schrieb, was Semái zu dem Zeichen ``Mein Herz sieht dich'' 
inspiriert hatte. \\
\textit{Ein Buch voller Gedichte mitten in der Wildnis}, dachte er melancholisch: 
\textit{Vielleicht werde ich ja selbst noch zum Poeten.}\\
``Bruder.''\\
Semái biss sich auf die Zunge um nicht zu fluchen. ``Mibell... ich dachte du schläfst schon.''\\
``Wenn du nicht aufpasst, bricht bald der Morgen an und alle sind wieder wach. Du gehst.''\\
Es war eine Feststellung, keine Frage. Deshalb antwortete Semái auch nichts. Wozu etwas sagen, wenn 
alles gesagt war? Wozu Worte verschwenden, wenn man sie für wichtigere Dinge aufheben konnte?\\
``Wolltest du dich noch bei mir verabschieden?'', fragte Mibell. \\
Sein Schweigen machte sie ungeduldig. So war es meistens mit den Menschen. Wenn er nichts sagte, 
sprach der Andere nur noch mehr. Und manchmal erzählte derjenige dann auch mehr als er eigentlich 
wollte. \\
``Nein'', antwortete er: ``Denn ich gebe dich nicht auf.''\\
``Du bist diesem Dichter etwas zu sehr zugetan, finde ich. Vermutlich bekomm ich gar nicht mit, wie 
viele Dinge die du sagst nur Zitate von Sovatis sind.'' Auch wenn ihr Tonfall spöttisch klang, 
schimmerten ihre Augen feucht. ``Du bist mein großer Bruder. Mein einziger Bruder. Und du willst 
mich in Stich lassen.''\\
``Ich komme wieder. Im Gegensatz zu Sovatis existiert meine Heimat. Sie ist da, wo meine Schwester 
ist'', log er und blickte sie eindringlich an.\\
``Dumm für mich, dass du auch eine andere Schwester nennst. Eine, die der Grund ist, wieso du 
gehst.''\\
``Nicht der Grund. Nur der Anstoß'', korrigierte er.\\
Mibell trat hastig auf ihn zu und legte ihre Hände auf seine Schultern. Eine Träne ran über ihre 
Wange und verharrte an ihrem Mundwinkel. ``Sie wird dich niemals so lieben wie ich es tue. Sonst 
würde sie dich nicht bitten, mit ihr zu kommen. Deshalb bitte ich dich nicht bei mir zu bleiben.''\\
Semái zögerte. ``Vielleicht solltest du mich bitten zu bleiben...''\\
Mibell ließ seine Schultern los und trat an das Fenster. Sie legte den Kopf in den Nacken um die 
schmale, hell erleuchtete Mondsichel betrachten zu können. Vorsichtig trat er neben seine Schwester 
und folgte ihrem Blick. ``Anta strahlt hell heute Nacht.''\\
``Anta und Kanto sollen dich auf deinem Weg begleiten, Bruder'', flüsterte Mibell, während weitere 
Tränen über ihr Gesicht liefen: ``Sag mir, wo euer Ziel ist, damit ich dich irgendwann wieder 
finden kann.''\\
``Das Ende der Welt?'', seufzte Semái: ``Und ich glaube, du bist momentan in einer größeren Gefahr 
als ich, wenn ich gehe. Der Senat intrigiert gegen unsere Familie.''\\
``Und auf dich wartet Gesindel, das gerne mit scharfen Klingen spielt. Und die Kreaturen... Wenn du 
kein guter Kämpfer wärst, würde ich dich nicht gehen lassen. Das weißt du, oder? Dann hätte ich dich 
längst nieder geschlagen und auf dem Bett gefesselt.''\\
``Das sind die romantischsten Worte die je eine Frau zu mir gesagt hat... und dann ausgerechnet von 
meiner Schwester!''\\
Sein Witz entlockte ihr nur ein trauriges Lächeln. \\
Mibell griff an ihren Gürtel und löste einen Knoten. Ihre Hand zitterte, als sie ihm die 
Dolchscheide reichte. Semái ergriff sie und zog andächtig die Klinge hervor. Sie war schlicht, aber 
das Mondlicht verlieh ihr ein sanftes Schillern. Ein Glassplitter war im Griff verarbeitet und 
somit trug die Klinge das Zeichen des Mondes. Antas Zeichen. Mibells Handfläche glitt über die 
Klinge und Semái tat es ihr nach. Der Dolch war so scharf, dass bereits wenig Druck ausreichte um 
durch seine Haut zu schneiden. Fest ergriffen die Geschwister die blutende Hand des anderen und 
sahen sich tief in die Augen. Augen, die einander glichen, aber deren Träume so unterschiedlich 
waren. \\
``Sollte dir irgendjemand etwas zu leide tun... werde ich kommen. Ich werde kommen um dir zu helfen 
oder dich zu rächen. Selbst wenn ich bis ans Ende der Welt reisen muss!''\\
Semái blickte ihr fest in die Augen, auch wenn er ihren Blick gerne ausgewichen wäre. Es war klar, 
dass dies nicht nur ein Versprechen war. Und es war offensichtlich, dass sie momentan Inu für die 
größte Gefahrenquelle hielt, die ihm bevorstand. Aber er würde sich nicht zu einer Erwiderung 
hinreisen lassen. Er würde sich nicht zwingen lassen, sich zu entscheiden. \\
\textit{Mibell teilt mein Blut. Wir teilen uns unsere Geschichte und unsere Ahnen. Mit Inu teile 
ich Hoffnung und Träume...}\\
Beide hatten einen festen Platz in seinem Herzen. Mibell hatte ihn geschenkt bekommen; Inu hatte 
ihn sich verdient. \\
``Nimm den Dolch. Damit du nicht vergisst, wer du bist, woher du kommst und wer an dich denkt.''\\
Semái nickte nur. Er sollte jetzt gehen, den Abschied nicht unnötig in die Länge ziehen. Aber noch 
konnte er sich nicht los reißen von den Anblick seiner kleinen Schwester, die ihn immer beschützt 
hatte. Eine andere Stimme brachte ihn jedoch dazu.\\
``Du bist mein Fehler.''\\
Semái wandte sich seiner Mutter zu, die im Türrahmen stand. Ihr langes Haar war zerzaust, ihre 
Wangen vom weinen gerötet. Er hatte die sonst starre Frau noch nie so aufgelöst gesehen. ``Es ist 
mein Fehler. Meine Schuld. Ich habe dir das Leben geschenkt. Es tut mir leid.''\\
Semái konnte sie nicht länger anschauen. Sein Griff schloss sich fest um den Dolch, damit sie das 
Zittern seiner Hand nicht sehen konnte. Das wäre nur ein weiteres Zeichen seiner Schwäche. \\
``Du versteht mich falsch'', antwortete sie: ``Ich trage die Schuld. Ich trage sie seit einem 
Jahrzehnt. Ich habe gebetet. Ich habe Opfer gebracht, aber die Götter straften mich. Dass sie mir 
meinen Sohn rauben ist die größte Strafe, die sie mir zufügen konnten.''\\
``Ich hatte nie die Gabe. Also hattest du auch nie einen Sohn'', murmelte Semái und berief sich auf 
den Kodex der Chanahe, der es erlaubte, Kinder ohne Gabe auszusortieren oder zu verbannen.\\
``Das ist ihre Strafe'', wiederholte seine Mutter: ``Aber nur weil Kanto und Anta mich verdammen, 
heißt das nicht, dass sie dich nicht geleiten, Semái. Meine Fehler sind zahlreich. Dein einziger 
war es, als mein Sohn geboren zu werden.''\\
Semái blickte zu Mibell, aber sie schien genauso ratlos zu sein wie er sich fühlte. \textit{Wovon 
spricht sie?}\\
Die Hände seiner Mutter griffen in ihren Nacken und lösten ihre Halskette. Mit ihrem schwebenden 
Gang trat sie näher und legte ihm die Kette um den Hals. ``Mibell gab dir ein Zeichen Antas. Ich 
gebe dir Kantos Zeichen. Mögen die Götter dich behüten und dich nicht verlieren.''\\
Der schwarze Kristall fühlte sich kühl auf seiner Haut an. Dagegen war der Kuss seiner Mutter warm. 
Semái konnte sich nicht entsinnen, sie je so bekümmert gesehen zu haben. ``Da mein Plan, heimlich 
aus dem Fenster zu steigen, gescheitert ist... kommt Vater auch?'' Seine Stimme klang 
hoffnungsvoll. Aber er brauchte keine Antwort abzuwarten, denn der Blick seiner Mutter genügte 
schon. \\
``Er sagte, er respektiert, dass du dein Bestes getan hast.''\\
\textit{Aber es hat nicht gereicht.}\\
Semái holte tief Luft. Er drückte Mibells Hand und auch ihr gehörte der letzte Blick, ehe er den 
Raum verließ. Zum letzten Mal trat er durch diesen Flur, diese Treppen hinunter und durch diese 
Türe. Im Salon brannte noch das Kaminfeuer. Der Schatten eines Mannes wurde an die Wand geworfen. 
\textit{Also ist er doch wach...}\\
Semái überlegte, ob er inne halten sollte. Aber er entschied sich dagegen und versuchte möglichst 
leise an der offenen Tür vorbei zu huschen. Nicht leise genug. \\
Die Stimme war ohne Wärme oder Trauer. Aber Semái kannte seinen Vater und 
wusste, welche Überwindung diese Worte ihm gekostet haben mussten. ``Nimm den Hund mit.''\\


Inu hatte die letzte Nacht in Arashad schlafend auf einem kleinen Mäuerchen verbracht. Das war auch 
der Treffpunkt, den sie mit Semái ausgemacht hatte und dort herumlungernd fand er sie auch vor. 
Sein dunkles Haar war zerzauster als sonst. Mit zusammen gekniffenen Augen versuchte sie zu 
erkennen, ob er geweint hatte. Sie hatte ihn noch nie weinen sehen und hätte ihn gerne damit 
aufgezogen. Aber nein, der junge Mann wirkte wie üblich, wenn er bekümmert war. Noch schweigsamer 
als sonst und der Blick in eine andere Welt gerichtet. \textit{Wenn du ihn wirklich nur 
akzeptierst, weil er kämpfen kann, wird er dir nicht so viel bringen. Er sieht nicht aus wie ein 
großer Krieger.}\\
\textbf{Wenn es zu einem Kampf kommt, wird er schon nützlich sein. Ansonsten brauchen wir halt noch 
andere.}\\
\textit{Andere?}\\
``Morgen'', murmelte er und lehnte sich gegen das Mäuerchen, auf dem Inu noch ausgestreckt lag. \\
Sie streckte sich wie eine Katze und gähnte ausgiebig. ``Morgen. Verdammt tut mein Rücken weh!''\\
Semái warf ihr einen verwunderten Blick zu. ``Wie lange liegst du hier schon? Warst du nicht 
zuhause?''\\
``Wieso sollte ich?''\\
``Ich weiß, du magst deine Familie nicht sehr... aber du willst weg gehen... für eine lange Zeit. 
Willst du ihnen nicht auf Wiedersehen sagen?'', fragte Semái.\\
Inu richtete sich auf und ließ ihre Beine in der Luft baumeln. ``Wohl eher Lebe wohl. Ich war seit 
Wochen nicht mehr dort, Semái. Mein Vater hat mich rausgeworfen. Zu viele Esser. Da ich nicht den 
Nachbarn heiraten wollte... war ich hier oder im Wald.''\\
Sie hatte ihm das nicht gesagt, weil sie genau wusste, wie erschüttert er darauf reagiert hätte. 
Vermutlich hätte er sie noch dazu überredet, nachts heimlich in seinem Zimmer zu schlafen. Was dann 
geschehen wäre, wenn seine Eltern sie erwischt hätten... Inu schüttelte stumm den Kopf. Jeden Tag 
sah man Lisoe auf dem großen Platz die für Nichtigkeiten ausgepeitscht wurden. Darauf hatte sie 
verzichten können. ``So kalt war es noch nicht'', versuchte sie ihn aufzuheitern, denn sein 
trauriger Blick vermieste ihre Stimmung.\\
\textit{Ihm geht's gar nicht darum, dass ich auf der Straße geschlafen habe... oder?}, fragte Inu 
Saráe, die meistens Menschen besser einschätzen konnte.\\
\textbf{Nein}, antwortete die Göttin sachlich:\textbf{Er bemitleidet dich, weil er denkt, dass du 
niemanden hast, der dich liebt.}\\
Inu schluckte schwer. Sie dachte an seine wütenden Worte am See. Ja, er hatte recht. Ihr Vater 
scheute nicht davor zurück, missgestaltete Kinder zu ertränken oder Kinder frühzeitig aus dem Haus 
zu werfen. Inu selbst hatte nur so lange bleiben dürfen, weil sie der Mutter mit den Kleinsten 
geholfen hatte. Diese Rolle würde nun eine andere Schwester übernommen haben, bis sie verheiratet 
war oder auch vor die Tür gesetzt wurde. Daher kamen doch die viele Bettler in der Stadt. 
Straßenkinder, Huren, Diebe. Die meisten waren kaum älter als 15. Inu sah sich auf der Straße um. 
Langsam erwachte der Tag und Licht viel auf die verschlafenen, dünnen Gestalten. \textit{Die 
verstoßenen Kinder der Lisoe.}\\
\textbf{Kanto und Anta sind schuld. Sie zwangen die Menschen in verschiedene Klassen. Daraus kann 
nur Elend erwachsen.}\\
\textit{Saraé? Er hat unrecht, oder? Du liebst mich...}\\
Das Amulett um ihren Hals erwärmte sich. \textbf{Mein Mädchen im Mondschein}, wisperte Saráe sanft, 
beantwortete aber nicht die Frage.\\
``Wollen wir los? Ich habe etwas Geld, wir können uns auf dem Markt noch ein paar Dinge kaufen'', 
schlug Semái vor.\\

Schon zu den frühen Morgenstunden herrschte auf dem Markt ein eifriges Treiben. Um diese Uhrzeit 
stapften die Lisoe über den Platz, während die Chanahe sich erst im späten Mittag oder frühen Abend 
einfanden. Sie liefen an mehreren Ständen vorbei. Während Semái darüber sprach, was sie gebrauchen 
könnten, sah Inu sich zum letzten Mal hier um. Sie sah die eingefallenen Gesichter der Lisoe. Die 
dreckigen Kinder, die fangen spielten. Den Mann, den sie gestern hier ausgepeitscht hatten. Er hing 
noch gefesselt dort am Pranger und Fliegen summten um seine offene Wunde herum. Anscheinend hatte 
er keine Verwandten, die ihn nachts heimlich befreit hatten. Sie bedauerte das, würde es aber jetzt 
niemals riskieren, ihm zu helfen. Man musste Verbrechen ja nicht vor solchen Menschenmengen 
begehen. Und erst recht nicht für einen Fremden. \\
``Ein Seil. Einen Kochtopf. Schüsseln kann ich uns schnitzen. Ein Schnitzmesser! Und du brauchst 
eine Decke'', zählte Semái auf.\\
Inu nickte nur und während er mit einem Mann verhandelte, sah sie etwas viel interessanteres. Eine 
blonde Frau - ganz eindeutig eine Chanahe. Sie tanzte. Viele Menschen führten Kunststücke vor um 
Geld zu erbetteln, besonders hier am Marktplatz. Aber Chanahe waren niemals dabei. Inu wusste von 
Semái, dass es Theater für die Begnadeten gab, in denen ihresgleichen musizierte, tanzte oder 
Dichtungen vortrugen. Sie selbst hatte keinen Zutritt zu einen dieser Theater erhalten und hatte 
bisher auch nie wirklich geglaubt, dass die Chanahe Interesse für andere Künste als den Arenakampf 
hatten. Skeptisch sah sie der blonden Frau zu. Sie trug leichte Kleidung, welche zwar blickdicht 
war, aber aus feinem Stoff zu bestehen schien. Der Stoff hatte eine blasse Fliederfarbe und ein 
Armband aus Rosenquarz schimmerte an ihrem Handgelenk. Ihr Gesicht war verschleiert mit dem selben 
Stoff aus dem ihre Kleidung gefertigt war. Nur ihre Augen, die selbst einen blassen violetten Ton 
hatten, erkannte man. \\
\textit{Eine Begnadete die sich versteckt?}\\
Um ihre Handgelenke und Fußgelenke waren blassblaue Tücher gebunden. Die Frau stand aufrecht, ihr 
Gewicht auf beiden Beinen ausbalanciert. Sie hob den rechten Arm in einer langsamen, sanften 
Bewegung und das Stofftuch schnitt durch die Luft. Eine kleine Menschenmenge hatte sich bereits um 
sie versammelt und auch die, die nur vorüber eilten, warfen ihr neugierige Blicke zu. 
Das war das Schlimmste mit den Chanahe. Sie sahen so schön aus, dass man kaum wegsehen konnte. 
Schöne, kalte Menschen bei denen manche Lisoe gerne glaubten, dass sie einer anderen Spezies 
angehörten. Aber Inu wusste es besser. Sie hatte viele Nächte mit Saráe darüber diskutiert und war 
zu dem Entschluss gekommen, dass auch die Lisoe so schön sein könnten, wenn sie ihre Tage nicht mit 
mühevoller Arbeit zwischen Dreck und Unrat verbringen müssten. Wenn sie genügend zu essen hätten 
und vor allem eine gesunde Ernährung. \\
Die verhüllte Chanahe vollführte eine weitere Bewegung mit ihrem anderen Arm. Die beiden Tücher 
zogen durch die Luft. Ruckartig hielten ihre Hände inne und sie streckte sie mit den ineinander 
verschlungenen Fingern hoch über ihren Kopf. Die Tücher formten eine bauschige Wolke aus blauem 
Samt die sich einen kurzen Moment lang wie eine Qualle zusammen zog und wieder ausbreitete. Und 
dann begann ihr Tanz. Inu hatte schon viele Chanahe in der Arena kämpfen sehen. Die am weitesten 
verbreitete Gabe war die des Fliegens, wie sie es als Kind genannt hatte. Es war nie fliegen 
gewesen. Eher klettern und springen. Aber das hier, was diese Frau da tat... wenn es nicht fliegen 
war, was dann? Noch nie hatte Inu eine Begnadete gesehen, die ihre Gabe für etwas so schönes 
einsetzte... und sie so perfekt meisterte. \\
Die Chanahe schraubte sich in den Himmel. Wirbelnde Bewegungen der Arme. Hände, die sich für den 
Augenblick eines einzelnen Herzschlages an ihren Tüchern festhielten, hochzogen und neuen Schwung 
in ihre Bewegungen brachte. Ihre bloßen Füße schienen auf den wabernden oder durch die Luft 
zischenden Tüchern Halt zu finden. Ihre Bewegungen gaben den Tüchern die Energie, während diese 
wiederum ihr den Halt gaben für erneute Drehungen und Sprünge. Sie wirbelte, streckte 
ihre Gliedmaßen von sich, wurde komplett von bauschigen Stoff der Tücher eingehüllt, welche im 
nächsten Moment wieder wie Schlangen durch die Luft sausten.\\
``Ich habe noch nie eine Chanahe gesehen, die so gut ist'', murmelte Inu leise.\\
Semái war neben ihr getreten und sah dem Tanz ebenso zu, wie nun jeder in der Nähe. Überall sah man 
nur staunende Gesichter. Bloß Semái sah ruhig und fast schon berechnend zu, folgte ihren exakten 
Bewegungen. ``Ich auch nicht'', erwiderte er: ``Sie ist der Inbegriff von Disziplin und Perfektion. 
Das muss Jahrelanges Training gewesen sein... und eine große Portion Talent.''\\
Inu verdrehte die Augen. So sehr sie auch eigentlich die Begnadeten verabscheute, empfand sie seine 
Sachlichkeit was diesen einzigartigen Tanz anging als nervig.\\
Der Tanz hatte seinen Höhepunkt erreicht. Die Frau wirbelte ein letztes Mal durch die Luft. 
Kopfüber zog sie über den Zuschauern vorbei. Die Tücher ballten sich zu einer letzten Wolke, dann 
sank sie sanft wie eine Feder hinab. Ihr Fuß ruhte auf der Spitze eines einzelnen Tuches, während 
sie mit dem Stoff zu Boden glitt. Sie stützte sich auf ein Knie am Boden auf, den Oberkörper 
vorgebeugt. Das Haar war verschwitzt, ihre Beine und Arme zitterten vor Anstrengung und 
Erschöpfung. Der Tribut für einen Tanz, der keine fünf Minuten gedauert hatte und doch das 
aufregendste gewesen ist, was Inu je gesehen hatte.\\
Die Lisoe begannen aufgeregt zu reden. Helles Klimpern erklang, als zahlreiche Münzen in ihre 
Schale geworfen wurden. Und doch wagte sich keiner näher an sie heran. Auch wenn sie gerade viele 
Menschen entzückte, war sie fremdartig. Eine Begnadete. Sie war anders als die anderen Künstler, 
die hier am Markt ihre Fähigkeiten zeigten. Sie schenkte den Leuten kaum Beachtung, verneigte sich 
auch nicht oder bedankte sich für die Spenden. Der völlige Kontrast zu den sonstigen 
Straßenkünstlern, die um Aufmerksamkeit und Zuneigung des Publikums buhlten wie ein hungriger 
Hofhund. Langsam begann die Menge sich aufzulösen und zu Inus Überraschung trat Semái lächelnd auf 
die Frau zu.\\
``Das erklärt einiges'', sprach er sie in einem vertrautem Tonfall an. \\
Ihr Blick war kühl und distanziert, während sie das Geld zählte und sich den Beutel um die Schulter 
hängte. ``Ach ja? Und was?''\\
``Nicht so misstrauisch'', tadelte Semái schmunzelnd: ``Ich erkenne deinen Stil wieder, auch wenn 
du dich verschleiert hast. Man vergisst nicht so schnell jemanden, der einen verprügelt hat.''\\
``Ich hörte, du wärst schon von einigen verprügelt worden, Semái Jel Malaza'', spottete sie. \\
``Verwandte zählen nicht'', sagte er und sah sich um: ``Also? Fremd in Arashad, erst seit wenigen 
Wochen hier. Du kämpfst in der Arena für dein Haus. Und dann stehst du am nächsten Tag hier, zu 
einer Zeit, in der kaum Chanahe unterwegs sind... und bettelst? Reichte das Preisgeld für deinen 
Sieg über mich nicht lange aus?''\\
``Es ist nicht für mich!'', fauchte sie und drückte den Beutel fest an sich: ``Und wenn, geht es 
dich auch nichts an!''\\
Inu sah von einem zum anderen und fragte sich, was sie verpasst hatte.\\
``Das ist es'', sagte er: ``Dass, was ich dich schon in der Arena fragen wollte... woher kommt dein 
Hass?''\\
Die beiden Frauen sahen Semái überrascht an. Inu kniff verärgert die Lippen zusammen. Sie war es 
gewöhnt, dass ihm viele Dinge und Emotionen auffielen, die ihr selbst entgingen. Viel zu oft konnte 
Semái sogar sicherer Inus Gefühle bestimmen als sie selbst. Diese Tatsache an sich duldete sie ja 
meistens noch, aber seine Angewohnheit, Fremde dann auch noch auf deren Gefühle anzusprechen, 
verärgerte sie. Es war eine Sache ob ihr Bester Freund sie fragte oder ob er jeden auf der Straße 
anspricht. Die Chanahe hatte offensichtlich auch nicht damit gerechnet und wirkte sprachlos.\\
``Du bist unhöflich, Semaí'', feixte Inu und grinste böse. Es tat gut ihm mal diesen Vorwurf 
machen zu können.\\
Doch er ignorierte sie und blickte die Fremde abwartend an. Das war auch so etwas an ihm. Wenn er 
einen nur lange genug so ansah, dann musste man einfach Reden. Oder wegrennen, das hatte auch 
einige Male geklappt. Inu schüttelte frustriert den Kopf und bemühte sich nicht, ihre Ungeduld zu 
verstecken. Die blonde Chanahe warf ihr einen misstrauischen Blick zu und antwortete: ``Die Frage 
ist eher, wieso teilst du nicht den selben Hass?''\\
``Ich hasse nicht.''\\
\textbf{Dieser junge Mann ist einfach so faszinierend... So viel Ehrlichkeit in seinen Worten... 
faszinierend.}\\
\textit{Du kannst ihn doch nicht leiden,} erinnerte Inu Saráe.\\
\textbf{Das hat ja nichts mit Faszination zu tun.}\\
``Sie verstoßen dich, grenzen dich aus'', sagte die Frau leise und sah sich unruhig um, ob jemand 
in der Nähe war, der ihre Worte hören könnte.\\
``Ja... aber dich nicht. Warum hasst du sie deswegen?'', fragte Semái.\\
Sie zögerte nur kurz. ``Warum bist du hier um diese Uhrzeit mit einer Lisoe? Und Gepäck? Sag bloß, 
deine Familie hat dich doch noch verdammt, obwohl du gestern für sie in die Arena bist?''\\
``Es ist mein Wille zu gehen'', antwortete er.\\
``Wohin?''\\
Semái warf Inu einen Blick zu und grinste. ``Zum Ende der Welt?''\\
Auch Inu lächelte flüchtig. Sie war zu aufgeregt was dieser neue Abschnitt ihres Lebens mit sich 
bringen würde um sich von dieser Chanahe ihre freudige Erwartung vermiesen zu lassen. ``Vielleicht 
auch bis zum Anfang, fügte sie scherzhaft hinzu.\\
Die violetten Augen weiteten sich. ``Das Imuril? Wieso?''\\
``Wieso nicht?'', entgegnete Inu frech und fügte spottend hinzu: ``Semái, dann kannst du Kanto und 
Anta gleich fragen, wieso sie dir deine Gabe nicht gewährten!''\\
Sarju sah zu Boden und überlegte fieberhaft. Ruckartig hob sie schließlich den Kopf und fragte: 
``Hättet ihr was gegen Gesellschaft? Wir wären auch keine Last. Die Wege sind gefährlich, die 
zwielichtigen Gestalten zahlreich. Die Kreaturen tödlich. Ich bin eine bessere Kämpferin als du, 
Semái. Und vermutlich besser als eine Lisoe je sein könnte.''\\
Es sollte keine Beleidigung sein, dafür war ihr Tonfall zu bittend. \textit{Genau das ist das 
Problem mit diesen Chanahe, sie merken nicht einmal, wenn sie sich für etwas besseres halten!}\\
\textbf{Aber sie hat recht.}\\
``Wir?'', wiederholte Semái jedoch bloß.\\
Ihre Stimme war ein leises Murmeln. ``Mein Bruder ist wie du... er besitzt keine Gabe. Und noch 
weniger Talent im Kampf. Eigentlich ist er in nichts besonders gut. Aber er ist mein Zwilling, ich 
liebe ihn. Mein Haus ist groß, sie zögern nicht, solche Leute zu verbannen. Sie haben ihn schon 
weggeschickt, Wochen bevor wir in Arashad angekommen sind. Er hatte keine Ahnung wohin und ich habe 
mich heimlich mit ihm getroffen, er hat versucht sich einen Unterschlupf im Wald zu bauen. Hat 
nicht geklappt. Vermutlich ist er momentan in den schlechteren Gegenden der Stadt, wo kein 
Mitglied unseres Hauses hin gehen würde. Das Geld ist für ihn, aber ihr könnt es haben, wenn wir 
euch ein Stück begleiten dürfen.''\\
``Wieso tust du das für ihn?'', fragte Inu. Sie selbst hätte nie ihr behagliches Leben für einen 
ihrer Geschwister aufgegeben. Für Semái vielleicht, aber auch eher aus Schuldgefühl, weil er es für 
sie ganz bestimmt tun würde. \\
\textbf{Er tut es gerade für dich, du blindes Mädchen}, neckte Saráe: \textbf{Sie sollen mitkommen. 
Noch eine weitere Klinge, die für mich kämpft.}\\
\textit{Für dich? Sie wissen nichts von dir!}\\
\textbf{Noch nicht.}\\
``Er ist mein Zwilling'', entschied Sarju.\\
``Zwei Herzen, eine Seele'', fügte Semái hinzu und die blonde Frau nickte zustimmend.\\
``Dann lasst aber diese bescheuerten Chanahe-Sprüche und Gedichte sein!'', murrte Inu: ``Wenn dein 
Bruder im Wald war, kennt er bestimmt den See... wir übernachten heute dort, morgen bei 
Sonnenaufgang brechen wir auf. Mit euch oder ohne euch!''\\


``Ist das dein ernst?'', rief Inu und funkelte den Hund böse an. \\
Das Tier lümmelte auf Semáis Decke und blickte unschuldig zu ihr auf. Semái saß gelassen neben dem 
großem, weißen Tier, welches Inu erst für einen Wolf gehalten hatte. Er kraulte den Hund am Kopf 
und hinter den Ohren. ``Ja. Sie heißt Marai. Und wenn du einfach zwei Fremde einladen kannst, dann 
kann ich auch meinen Hund mitbringen.''\\
``Du hast die Chanahe doch angesprochen! Und woher kommt der überhaupt? Du hast nie einen Hund 
dabei gehabt.''\\
``Sie gehörte meinem Vater.''\\
Inu verstummte und biss sich auf die Zunge. Resigniert schüttelte sie den Kopf. ``Ich dachte, du 
würdest die beiden gerne dabei haben.''\\
Semái legte den Kopf schief. ``Darüber hatte ich noch nicht nachgedacht. Aber ich hätte nicht 
erwartet, dass du mit noch mehr Chanahe reisen willst.''\\
\textit{Will ich auch nicht.}\\
``Und wir haben einen Tag verschenkt. Wir hätten schon längst das Tal verlassen'', fügte er hinzu 
und streckte sich auf der Decke aus. Sonnenlicht drang durch die Baumkronen und tanzte über ihn 
hinweg. Inu betrachtete ihn einen Moment, wie er friedlich neben dem Hund lag und seufzte. Sie 
würde ihm gerne die Wahrheit sagen. Nicht sie wollte die Chanahe dabei haben, sondern Saraé. Aber 
wie sollte sie Semái von der Existenz der Göttin erzählen? Er würde sie für verrückt halten. \\
\textbf{Oder schlimmer. Er würde dir glauben.}\\
\textit{Und wieso ist das schlimm?}\\
\textbf{Weil er ein guter Mensch ist. Viel zu gut für uns}\\
``Ich weiß nur gerne mit wem ich in die Wildnis gehe. Ich weiß nur gerne, wer nachts neben mir 
schläft, wer über das Feuer wacht, wer mir Rückendeckung gibt'', erklärte Semái.\\
``Du klingst, als würden wir in den Krieg ziehen.''\\
Er stützte sich auf die Ellbogen und blickte sie kritisch an. ``Inu... außerhalb diesen Tals, 
außerhalb der Reichweite der Chanahe... es gibt dort keine Regeln. Wir leben in der Welt des Chaos. 
Jeder Mensch handelt für sein eigenes Wohl, nach seinem eigenem Recht.''\\
``Ja, wir leben in dieser Welt. Freiheit und Chaos fließt durch unsere Adern. Und das ist auch gut 
so! Ich scheiße auf die Stadt und ihre Regeln!''\\
Semái nickte nur.\\

Die Zwillinge stießen bereits am Abend zu ihnen. Diesmal trug Sarju eine ähnliche Montur wie in der 
Arena. An ihrem Gürtel baumelte eine Schwertscheide und Inu konnte einen weiteren Schwertgriff über 
Sarjus Schulter erkennen. Sie hob ihre Hand zum Gruß, lächelte jedoch nicht. Hinter der blonden 
Frau tauchte ein weiterer Haarschopf auf. Sein Haar war dunkler und er schien es selbst kürzlich 
mit einem Messer geschnitten zu haben. Die Haarsträhnen standen teilweise in verschiedenen Längen 
von seinem Kopf ab. Er war ebenso groß wie seine Schwester, aber seinen Bewegungen fehlte ihre 
Eleganz und Leichtigkeit. Staub klebte in seinem Haar, Kratzer zierten seine Wange. Inu grinste. 
Sie lebte auch schon eine Weile auf der Straße, aber so gerupft sah sie doch noch nicht aus. \\
``Mein Bruder Revo'', erklärte Sarju und deutete auf den Mann hinter ihr. Er hob fast schon 
schüchtern den Kopf und musterte die beiden.\\
Es war Marai, die die beklommene Stimmung durchbrach. Wedelnd sprang der Hund auf die Beine, tapste 
auf die beiden zu und lief schnüffelnd um sie herum. Sarju ging lachend in die Knie und zauste 
durch das weiße Fell. Auch Revo rang sich ein scheues Lächeln ab und streckte seine Hand aus. 
Zutraulich schleckte sie über seine Finger und kehrte dann hechelnd an Semáis Seite zurück. \\
``Also Revo. Das ist der Kerl, den ich in der Arena fertig gemacht habe. Ohne meine Gabe 
einzusetzen!'', prahlte Sarju.\\
``Sie hat sich bemüht'', fügte Semái hinzu.\\
``Sieht der Abend jetzt so aus, dass die beiden sich ausweinen, wie schlimm es ist, ohne eine Gabe 
geboren zu sein?'', fragte Inu bissig: ``Dann such ich mir nämlich einen anderen Schlafplatz.''\\
``Und das ist eine zickige Lisoe. Den Grund ihrer Anwesenheit weiß ich auch noch nicht'', konterte 
Sarju.\\
``Sie ist meine Schwester'', erklärte Semái.\\
In seinem Tonfall lag etwas sachliches, was Inu sonst eher von Saráe gewohnt war. Keinerlei Emotion 
war in seinen Worten mitgefallen, als hätte er eine offensichtliche Tatsache ausgesprochen. Inu 
lächelte ihm flüchtig zu. Sie hätte nie gedacht, dass es sich so gut anfühlen kann, dass diese 
Worte jemand zu ihr sagt. Er war ihr so viel mehr ein Bruder als all ihre leiblichen Geschwister es 
hätten sein können. Sarju neigte in einer förmlichen Geste den Kopf und legte ihre rechte Hand auf 
ihr Herz. ``Verzei mir, dass ich unhöflich war.''\\
``Inu hat angefangen'', antwortete Semái und zuckte mit den Schultern: ``Ist schon in Ordnung.''\\
Revo lachte plötzlich laut los. ``Du bist wirklich nicht wie die anderen Chanahe. Für eine solche 
Beleidigung der... Schwester, hätten viele andere längst zum Duell gefordert.''\\
``Er weiß, dass ich wieder siegen würde,'' neckte Sarju und deutete auf den provisorischen 
Lagerplatz: ``Wie sieht es aus? Ein Feuer können wir uns gönnen, oder? Revo ist ganz gut darin, mit 
dem Bogen Hasen zu schießen. Und ich habe noch ein paar Andenken von meiner werten Familie mitgehen 
lassen.'' Sarju zwinkerte ihnen zu und fischte zwei tönerne Flaschen aus ihrem Beutel.\\

Am Abend kühlte die Temperatur nur wenig ab. Trotz der Mondsichel am Himmel drang kaum Licht durch 
die Blätter der Bäume über ihnen. Die Geräusche der Tiere hatten längst eingesetzt und auch die 
Bewegungen des Seewassers schienen bei Nacht lauter zu werden. Die vier jungen Menschen saßen auf 
Decken um ein hell prasselndes Feuer herum. Marai drückte sich eng an ihren Herrn und auch wenn sie 
bereits ihre Augen geschlossen hatte, blieben die Ohren gespitzt. \\
``Und dann...'', setzte Sarju zur Pointe ihrer Geschichte an: ``Packte ich den kleinen Idioten, 
kletterte auf den Turm und hängte ihn an seinen Hosen an die Turmspitze!''\\
Inu lachte schallend los, während Semái nur wie üblich grinste. Sogar Revo, der die Geschichte schon 
dutzende Male gehört und vermutlich sogar dabei gewesen war, lachte mit. Sarju kicherte hell vor 
sich hin, nahm einen weiteren tiefen Schluck von dem Schnaps und reichte die Flasche an Inu 
weiter. ``Jetzt seid ihr dran!''\\
Semái schnappte Inu die Flasche weg und nippte daran. Dann erzählte er: ``Hier gab's vor ein paar 
Jahren eine Pferdeherde. Sie tranken hier oft am See und zogen dann zu einer Lichtung oder zu den 
Wiesen in der Nähe von Inus Siedlung.''\\
``Und du hast versucht den Leithengst zu zähmen?'', riet Revo.\\
Grinsend schüttelte Semái den Kopf. ``Nein. Da war eine Stute. Sie war noch jung und ich 
beobachtete sie lange. Ihr Fell war fast so weiß wie das von Marai.'' Liebevoll kraulte er den 
Hund. ``Ich versuchte immer Näher an sie heran zu kommen, bis ich sie schließlich berühren konnte. 
Und dann... tja...''\\
``Versuchte er auf ihren Rücken zu klettern!'', rief Inu lachend aus und schüttelte den Kopf: ``Ihr 
hättet das sehen müssen! Er hing an ihrer Seite, hatte zu wenig Schwung und das Pferd lief los. Ein 
paar Schritte ist er noch neben her gelaufen... dann hat sie ihn über den Boden geschliffen, bis er 
losgelassen hat.''\\
Sarju schüttelte amüsiert den Kopf. ``Wie kommst du auf die Idee, auf einem Pferd sitzen zu wollen? 
Darauf wäre ich nicht mal gekommen... und ich habe einigen Mist gebaut. Pferde sind für den Acker 
da um Pflüge zu ziehen. und auch das können Ochsen besser.''\\
Semái beugte sich verschwörerisch vor. ``Aber was wäre, wenn man ein Pferd wirklich dazu bringen 
würde? Stellt euch mal vor, wie schnell man reisen könnte! Oder fliehen... Ein Begnadeter würde es 
bestimmt schaffen.''\\
``Sich drauf festzuhalten vielleicht. Aber deshalb macht das Tier noch lange nicht das, was ich ihm 
sage.''\\
Inu überlegte. ``Aber dem Ackergaul hat man ja auch beigebracht dort hin zu gehen, wo man es ihm 
sagt.''\\
``In dem man den Gaul mit Futter lockt oder mit der Peitsche droht'', erwiderte Revo.\\
Inu deutete auf den Hund. ``Er kommt auch immer zu Semái zurück! Er hört auf das, was er ihm 
sagt.''\\
``Sie'', korrigierte Semái und reichte die Schnapsflasche weiter im Kreis.\\
Eine Weile herrschte nachdenkliches Schweigen. Dann fragte Sarju: ``Ist das eigentlich euer 
ernst? Das Ende der Welt?''\\
Inu fühlte sich angesprochen, da es ihre Idee gewesen war, und zuckte mit den Schultern. ``Oder den 
Anfang. Wenn es einen Anfang gibt, gibt es auch ein Ende, oder?''\\
``Es heißt, das Imuril ist der Ursprung. Die Hölle. Dort lauern Kreaturen, die wir uns nicht 
vorstellen können. Und sie sind nur dafür da, Menschen die die Tore durchqueren zu töten'', 
murmelte Revo.\\
Seine Schwester sagte: ``Nun... dann würde ich wirklich erst mal das Ende anstreben. Wenn der 
Anfang die Hölle ist, kann das Ende nur besser werden, oder?'' Sie lachte wieder und begann ihr 
blondes Haar mit den Fingern zu kämmen. ``Norden, Süden, Westen oder Osten?''\\
``Warum nicht Südwest? Oder Nordost?'', warf Semái ein.\\
``Oh, nun musst du auch noch alle anderen Kombinationen nennen, sonst fühlen sie sich 
ausgegrenzt'', spottete Inu.\\
``Wir sind aus Südost gekommen...'', sagte Sarju: ``Diese Reise soll meine Zukunft sein und mich 
nicht in meine Vergangenheit zurück führen. Und ich kann garantieren, dass da ein langes Stück kein 
Ende der Welt in Sicht ist.''\\
Revo nickte zustimmend. Tief holte Semái Luft. ``Verstehe. Ich glaube in keiner Richtung ist das 
Ende oder der Anfang sehr nahe... aber ich will nicht in südwestliche Richtung aufbrechen. Aus dem 
selben Grund.''\\
Inu runzelte die Stirn und bedauerte, dass die Stimmung plötzlich so viel bekümmerte geworden ist. 
\textit{Saráe... wohin sollen wir gehen?}\\
\textbf{Nach dem Ausschlussverfahren... dorthin, wo Kanto sich nicht hintraut.}\\
``Dann in den Norden, wenn ihr alle sonst in Tränen ausbrecht'', entschied Inu spöttisch und 
schnappte sich das letzte Stück des gerösteten Brotes. \\





