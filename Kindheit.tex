\chapter{Kindheit}


So leise wie möglich schlich sie über den Lehmboden des kleinen Hofes. Zaghaft setzte sie einen 
Fuß nach dem anderen auf, vermied jedes Ästchen oder dürres Blatt. Dem alten Hofhund am Tor entging 
das aber alles nicht. Aus alterstrüben Augen beobachtete er ihr Bemühen, wedelte einmal kurz mit 
seinem buschigen Schwanz und legte seufzend den Kopf zurück auf seine Pfoten.Nachdem Inu das Tor 
durchschritten hatte, wurde sie mutiger und schneller. Sie lief an den windschiefen Zäunen der 
anderen Höfe vorbei, sprang über einen leise plätschernden Bach und begann, ohne groß nachzudenken, 
zu rennen. Sie kannte die Strecke über die Wiese gut, daher reichte das Mondlicht um sich zu 
orientieren. Das trockene Gras strich über ihre nackten Knöchel hinweg, kratzte und stach in die 
Fußsohlen. Aber das alles störte Inu nicht. Ihr Rock wehte um ihre dürren Beine, das braune Haar 
hinter ihr her. Sie rannte so schnell, wie sie sonst nur lief, wenn einer der wütenden Bauernjungen 
hinter ihr her war. Ihre Lunge brannte und etwas stach schmerzhaft in ihre Seite, als Inu endlich 
stehen blieb. Sie ließ sich auf einen umgestürzten Baumstamm nieder und hielt die Hände vor dem 
Mund. Das Kichern war trotzdem hörbar. Es bahnte sich leise einen Weg hinauf und brach plötzlich 
wie 
ein Gewitterblitz über ihre Lippen.\\
\textbf{Das Mädchen im Mondschein lacht.}\\
Inus Glucksen und Kichern durchbrach die Stille der Nacht. Nicht einmal die Blätter in den 
Baumkronen raschelten. Selbst die dürren Gräser waren still, von Tieren war weder etwas zu hören 
noch zu sehen. Der Mond schien unablässig auf das kleine, dürre Mädchen, welches kichernd auf dem 
morschen Baumstamm saß.\\
\textbf{Warum lacht das Mädchen im Mondschein?}\\
``Weil es lustig ist!''\\
\textbf{Was findest du lustig?}\\
Inu stockte und begann zu überlegen. ``Weiß nicht..:''\\
Sie sah sich um, konnte aber den Ursprung der Stimme nicht erkennen. Inu kniff die Augen fest 
zusammen. ``Wer bist du?''\\
\textbf{Wer bin ich?}\\
Inu begriff, dass waren keine Worte, zumindest nicht solche wie sie kannte. Es war anders... die 
sanfte, leiste Stimme huschte wie ein Windhauch durch ihren Kopf. Inu stand auf und tastete sich 
vorsichtig heran.  Sie ergriff das dünne Lederband und hob es hoch. Der Anhänger baumelte direkt 
vor ihrem Gesicht in der Luft. ``Wie heißt du?'', flüsterte Inu neugierig.\\
\textbf{Saraé}\\
Eingehend betrachtete Inu das Amulett. Auf der hellen Elfenbeinscheibe war ein schwarzes, 
verschlungenes Muster zu sehen. Vielleicht war es eingebrannt, aber Inu konnte das nicht genau 
sagen. Immerhin hatte sie in ihrem jungen Leben noch kein einziges Schmuckstück in den Händen 
gehalten. Unwillkürlich dachte sie an ihre aufgebrachten und besorgten Eltern. Wenn sie bemerkten, 
dass Inu sich fort geschlichen hatte, würden sie bestimmt höllisch wütend werden.\\
\textbf{Nimmst du mich mit?}\\
``Aber nur, wenn du leise bist!''\\
Glockenhelles Lachen. \textbf{Versprochen, Mädchen im Mondschein!}\\


Die nächsten Tage verbrachte Inu oft stundenlang versunken in Gesprächen mit Saraé. Das Mädchen 
trug das Amulett unter ihrem groben Wollhemd versteckt, damit die Eltern nicht schimpfen und die 
Geschwister es nicht klauen würden. Somit ließ sie Saraé an jeder Minute des Tages am Geschehen 
teilhaben. Die leise Stimme erzählte fantastische Dinge über fremde Welten, böse Dämonen und den 
Lichtern des Lebens. Inu begriff die Zusammenhänge nicht immer, aber sie träumte sich in diese 
Welten und lauschte entzückt Saráes Geheimnissen.\\
Am Rande bekam sie mit, dass die Erwachsenen immer zorniger wurden. Aber auch leiser. Sie schrien 
die Flüche über die Chanahe nicht mehr lautstark heraus, sondern es glich mehr dem leisen Zischen 
einer Schlange. Mit einem lauten Scheppern zerbarst die Tonschale auf dem Boden.\\
``Inu!'', schimpfte ihre Mutter: ``Was hast du schon wieder gemacht? Ich kann das langsam nicht 
mehr aushalten! Verschwinde!''\\
``Aber... Wohin denn?''\\
``Heram! Nimm Inu mit! Sie bringt hier alles nur durcheinander!''\\
Inu blickte mit großen Augen zur Tür. Ihr Vater steckte den Kopf herein, seine Gesichtszüge waren 
ebenfalls schon von schlechter Laune geprägt.\\
``Zur Hölle, was soll dieser Schwachsinn hier? Warum soll ich das Gör mitnehmen? Die steht doch nur 
unnütz im Weg herum!''\\
Und so ging das Streitgespräch, welches sich hauptsächlich um Inus Unfähigkeit drehte, weiter. Das 
Mädchen fühlte sich immer kleiner und wäre am Liebsten im Boden versunken.\\
\textbf{Du bist ein starkes Mädchen. Ich weiß das und du weißt es auch.}\\
\textit{Sie hassen mich}, dachte Inu und wünschte sich in diesem Moment nichts mehr, als eine 
Umarmung.\\
\textbf{Nein... Deine Eltern sind müde. Deine Eltern haben Angst.}\\
\textit{Wovor?}\\
\textbf{Der Zukunft.}\\
\textit{Muss ich auch Angst haben?}\\
\textbf{Ich werde immer bei dir sein. Dich beschützen. Wir sind die Zukunft.}\\
``Bei Kanto, auf was wartest du denn jetzt noch? Bewegung!'', herrschte der Vater.\\
Inu zuckte zusammen, folgte ihm aber eilig aus der Tür. Wie das Streitgespräch ausgegangen war 
hatte sie nicht mitbekommen. Aber ihr Vater drückte ihr den Strick der Ziege in die Hand. \\



``Du wirst aufpassen, dass sie in der Stadt keiner klaut. Wenn aber jemand sie kaufen will, dann 
sagst du, dass ich gleich wieder komme und mit ihm verhandeln kann, verstanden?''\\
Sie nickte schwach und der Vater marschierte ohne noch einmal einen Blick hinter sich zu werfen 
los. Inu brauchte einen Moment, um die verschlafene Ziege zum laufen zu bewegen und zerrte 
unschlüssig an dem Strick. Als sie endlich vorwärts kamen, war der Vater schon längst vom Hof 
verschwunden. Nur mühsam holte sie ihn ein und nachdem sie einige Schritte schweigend zurückgelegt 
haben, wagte Inu zu fragen: ``Warum bist du wütend auf die Stadt und die Chanahe?''\\
``Sie sind arrogante Idioten die meinen, sie dürften sich alles erlauben'', knurrte ihr Vater: 
``Außerdem meinen sie, all das Land gehört ihnen!''\\
``Aber es gehört doch niemanden... auch uns nicht. Das Land ist frei'', murmelte Inu schüchtern und 
dachte an die Geschichten über den Gott Kanto, der ihre Welt erschaffen hat. Eine Welt der 
Freiheit.\\
Ihr Vater spuckte aus. ``Und was machen die Chanahe? Bauen eine Stadt! Eine Stadt, ist das völlige 
Gegenteil von Freiheit! Eine Stadt ist eingrenzend, dreckig. Wie die Weiden des Viehs! Aber gut 
jetzt, du verstehst das eh noch nicht!''\\
Und noch weniger verstand Inu, warum ihr Vater in die 
Stadt, die er so hasste, ging um sich als Bauarbeiter zu bewerben.\\


Inu hatte noch nie eine Stadt gesehen. Und eigentlich konnte man die Ansiedlung bis jetzt auch noch 
nicht als eine bezeichnen. Trotzdem stand das Mädchen mit offenem Mund da. Die Siedlung umfasste 
bisher fünfzehn fertige Steinhäuser, die so groß waren wie der Hof, auf dem Inu mit ihrer Familie 
lebte. Zu den fertigen Häusern kamen aber auch mindestens zwei Dutzend Baustellen hinzu. Soweit Inu 
wusste, lebten die Chanahe mit den fertigen Häusern schon ein paar Jahre hier in der Nähe ihres 
Hofes. Und hier sollte anscheinend diese Stadt entstehen, vor der die Erwachsenen solche Angst 
hatten. \\
\textbf{Wer sind die Chanahe?}, fragte Saráe. \\
\textit{Das sind andere Menschen, versuchte Inu es zu erklären. Sie mögen uns nicht. Papa sagt, sie 
sind arrogant und denken, sie wären besser. Sie können... sie sind die Erwählten der Götter. Sie 
können Antas und Kantos Gaben benutzen.}\\
\textbf{Gaben}, wiederholte die Stimme nachdenklich.\\
\textit{Ja! Zum Beispiel... sehr gut klettern. Oder auf Wasser gehen. Oder ganz viele Sprachen 
sprechen. Aber auch fliegen.}\\
\textbf{Fliegen?}\\
Saraé klang nicht sehr überzeugt. \\
\textit{So ähnlich}, verbesserte Inu: \textit{Sie können auf den kleinsten Dingen springen. Und von 
einem fallenden Blatt auf das nächste hüpfen!}\\
Sie wurde aus dem Gedankengespräch mit Saraé gerissen, als die Ziege einen schrillen Laut ausstieß 
und einen Satz zur Seite machte. Vor Schreck ließ Inu das Seil los und die Ziege verschwand laut 
meckernd in der Schar arbeitender Männer. ``Hey! Bleib hier!''\\
Inu hatte erst einen Schritt in die Richtung der fliehenden Ziege getan, als ein weiterer Stein auf 
sie zu schoss und nur knapp an ihrem Ellbogen entlang schrammte. \\
``Was soll das?'', rief sie laut, duckte sich weg und hob ihre Arme schützend vor das Gesicht.
Ein Stein traf sie an der Wade, aber Inu biss sich fest auf die Zunge und kein Schmerzenslaut kam 
über ihre Lippen. Stattdessen fixierte sie aus funkelnden blauen Augen die Angreifer. Es waren drei 
Jungen, älter als sie, hochgewachsen und mit einer Ausstrahlung von purer Arroganz. \\
``Was machst du dreckige Lisoe hier?'', keifte der Älteste und klaubte einen weiteren, kantigen 
Stein vom Boden auf. \\
``Verschwinde!'', brüllte der Jüngste von ihnen, so aufgebracht, dass Inu seine Spucke beim 
sprechen fliegen sah.\\
``Was hab ich denn getan?'', fragte sie zurück.\\
``Lisoe sollen nur in Arashad sein, wenn sie arbeiten! Ansonsten sollen sie in ihre dreckigen 
Hütten verschwinden und uns Chanahe nicht im Weg stehen!''\\
``Und du Miststück hast offensichtlich nichts zu tun, also hau endlich ab!''\\
Ein Stein traf Inu am Knöchel, während sie einem höher fliegenden Geschoss auswich. Diesmal konnte 
sie einen Laut nicht unterdrücken. Wimmernd wich sie zurück und sah sich panisch um, wo ihr Vater 
nur war. Sie hatte schreckliche Angst vor den Jungen. So einen Zorn und Hass in den Augen hatte sie 
bisher höchsten mal bei Erwachsenen gesehen, aber der war nie gegen sie gerichtet gewesen.\\
\textbf{Du bist stärker, Inu.}\\
\textit{Nein, bin ich nicht}, erwiderte sie Saraé stumm. \textit{Ich bin kleiner und jünger und sie 
sind zu dritt!}\\
\textbf{Du hast mich. Kein Mensch kann je stärker sein als ich. Und ich schenke dir meine Stärke!}\\
``Hört auf!'', schrie Inu, mutig geworden durch Saráes ruhigen Worte: ``Lasst mich ja in Ruhe!''\\


Inu ergriff einen Stein. Er fühlte sich schwer und rau in ihrer Hand an, aber das ließ sie keinen 
Moment zögern, und sie schleuderte ihn durch die Luft. Der Stein traf den Jüngsten an der Stirn und 
er brach in lautes Heulen aus. In Rekordzeit bedeckte Rotz und Wasser sein Gesicht und er erinnerte 
Inu an das erbärmliche Geschrei eines Säuglings. Inu richtete sich auf und ballte die Hände zu 
Fäusten. Ein Windhauch ließ ihr braunes Haar tanzen und sie fühlte sich groß und stark wie nie.
``Will noch jemand was?'', herrschte sie die Knaben an. \\
Der Größte machte erst einen Schritt auf sie zu, dann rannte er. Er ergriff sogar eine 
herumliegende 
Holzleiste von der Baustelle und schwang sie drohend durch die Luft.\\
\textbf{Wir weichen nicht.}\\
Saráes Stimme klang so ruhig, so standhaft, dass Inu diese Ausstrahlung in sich aufsog. Dem Mädchen 
kam es so vor, als ob Saraé neben ihr stehen würde, so nah fühlte sie sich an. Irgendwas sah der 
angreifende Junge wohl in ihrem Gesicht, denn er zögerte. Dann kniff er die Augen wütend zusammen 
und beschleunigte sein Tempo.\\
\textit{Seine Gabe}, dachte Inu: \textit{Er will seine Gabe einsetzten!}\\
Der Junge rammte den Stock in den Boden und wollte wohl den Schwung nutzen, um sich in die Luft zu 
schwingen und mit den Füßen voraus auf sie nieder zu rammen. Die erste Hälfte seines Vorhabens 
gelang ihm auch. Er hob federleicht vom Boden ab, wie es wohl nur ein Begnadeter konnte. Auf seinem 
Gesicht zeichnete sich eine hässliche Grimasse der Überlegenheit ab. Er änderte seinen Schwerpunkt 
und brachte die Füße voraus. Und viel wie ein Stein zu Boden. Dort blieb er benommen sitzen und 
starrte Inu geschockt an. \\
``Hexe!'', flüsterte er. \\
Inu blickte zu den übrigen Jungen und sah, wie diese schon um die nächste Baustellenecke flohen. 
Das er nun im Stich gelassen wurde, stellte auch der Älteste fest, packte seinen Stock und rannte 
so schnell ihn seine Füße trugen ihnen nach. Einmal wandte er sich noch zu ihr um, stolperte dabei 
über seine Füße und wirkte, als hätte er einen Dämon gesehen. \\
\textit{Was war das?}, fragte Inu Saraé.\\
\textbf{Was sind Begnadete schon ohne ihre Gabe? Laufende Welpen, zitternd vor Angst. Die Chanahe 
sind wie eine Waagschale. Nimmt man ihnen ihre Gabe, raubt man ihnen das Gleichgewicht. Für diesen 
Moment schwanken sie, taumeln, fallen.}\\
``Meinst du, das war klug?'', erklang plötzlich eine völlig andere Stimme. \\
Inu wand sich dem Sprecher zu und kniff die Augen zusammen. Das Mädchen wartete auf weitere 
Beleidigungen, denn dieser Junge war ebenso ein Chanahe wie die drei anderen. Das erkannte sie 
schon an seiner Stimme. Er sprach anders, als die Bewohner des Tals. Die Betonung der Silben war 
verkehrt, einzelne Laute klangen bei ihm klar, wobei die Einwohner diese Laute verschluckten. 
Jedoch lag kein Hass in seinem Blick. Eher Neugierde und Nachdenklichkeit. Er war so groß wie Inu, 
obwohl er bestimmt auch etwas älter war. Seine Kleidung bestand aus robustem, aber gut 
verarbeitetem 
Stoff und sein dunkelbraunes Haar war zerzaust, seine Wangen gerötet. \\
``Lass mich in Ruhe!'', fauchte Inu.\\
Der Junge betrachtete sie und stieß dann einen der Steine mit dem Fuß weg. \\
``Was hättest du denn getan?'', rief sie ihm wütend und gekränkt zu.\\
Er zuckte mit den Schultern. ``Weiß nicht... aber Meroka Horusan zu verletzten war bestimmt nicht 
gut.''\\
``Wieso?''\\
Er runzelte die Stirn. ``Die Horusan sind eine der drei Familien, die diese Stadt errichten lassen. 
Naja, es sind mittlerweile schon mehrere dazu gekommen, aber die Horusan finanzieren ziemlich viel. 
Hoffe, dass du keinen Ärger bekommst. Wie heißt du?''\\
Inu zögerte. Der Junge schien nett. Immerhin hatte er sie noch nicht verletzt oder beschimpft. Auch 
wenn er ein Chanahe war. Sie dachte an die Wut ihres Vaters. War diese Wut vielleicht auch nur 
Angst? Hat ihr Vater Angst vor den erwachsenen Chanahe wie sie vor deren Kindern? \\
``Warum haben sie das getan?'', fragte sie leise. Ihre Stimme zitterte. Inu konnte diesen Hass auf 
sie nicht verstehen. Sie hatte den Jungen nichts getan. Sie kannte sie ja nicht einmal!\\
``Weil du eine Lisoe bist? Haben sie doch gesagt.''\\
``Was ist das?''\\
Der Junge sah sie groß an. ``Das weißt du nicht?''\\
Inu schüttelte den Kopf, hin und her gerissen zwischen den Gefühlen. Einerseits war sie wütend auf 
den Jungen, weil er sie so dumm aussehen ließ, andererseits sehnte sie sich nach netten Worten und 
Freundlichkeit. \\
``Komm mit'', erklärte er: ``Hier ist es mir zu laut.''\\

Inu warf kleine Kieselsteine den Hang hinunter und sah zu, wie diese weitere Steinchen auf ihren 
Weg ins Rollen brachte und alle schließlich im Flussufer untergingen. Sie hatte sich mit dem 
Jungen, als Semái Malaka hatte er sich vorgestellt, an den Abhang gesetzt. Vor ihnen lag nur Geröll 
und weit darunter rauschte der Fluss dahin. Die Baustelle hatten sie hinter sich gelassen, aber man 
konnte die Rufe der Arbeiter noch hören und sah die Umrisse der entstehenden Gebäude. \\
``Chanahe bedeutet, von den Göttern begnadet'', erklärte Semái ihr und warf einen Kieselstein. Er 
versuchte, von seiner Position aus, den Fluss zu erreichen. Aber die etlichen Versuche waren 
bereits alle vergebens gewesen und Inu konnte nur über sein weiteres Bemühen lachen. \\
``Begnadet?''\\
``Ja... also, die Gaben. Davon weißt du doch, oder?''\\
``Klar...'', sagte Inu selbstbewusst, auch wenn ihr Wissen nur vage war. Aber das konnte sie sich 
ja nicht anmerken lassen. Denn dann würde Semái sie für dumm halten und wieder gehen. \\
``Mein Vater kann im Dunkel sehen'', erzählte er: ``Also... nicht richtig. Aber er spürt das Holz 
des Bodens, das Gestein der Wände und die Erde. Deswegen kann man sich auch nicht an ihn 
heranschleichen! Und meine Mutter, die ist schnell wie der Wind!'' Er verzog das Gesicht: ``Ist 
schlimm... bin einmal abgehauen, weil ich keine Lust mehr auf das Lernen hatte. Aber die haben mich 
sofort gefunden und auch eingeholt.''\\
Inu kicherte und stellte sich vor, wie Semái vor seiner schnellen Mutter weg rannte. \\
``Und meine Schwester kann jetzt schon klettern wie ein Eichhörnchen und auf den dünnsten Zweigen 
stehen. Und sie ist erst zwei!''\\
``Könnt ihr alles? Auch fliegen? Andere in Tiere verwandeln?''\\
Er lachte. ``Nein! Vater sagt immer, die Magie der Götter fließt durch die Elemente... also meine 
Mutter ist schnell wie der Wind. Mein Vater lauscht der Erde. Und meine Schwester... wird wohl auch 
was mit Wind zu tun haben. Aber du wolltest ja wissen, was ein Lisoe ist. Das ist einfach. Das sind 
alle anderen.''\\
``Alle anderen'', wiederholte Inu: ``Aber dann ist es doch völlig normal. Dann bin ich völlig 
normal und ihr seid anders!''\\
Er zögerte. ``Naja... manche halten sich für etwas Besseres, weil wir von den Göttern begnadet 
sind.''\\
``Aber warum ihr und wir nicht?''\\
Semái schwieg darauf und warf nur den nächsten Stein.\\
\textbf{Ungerechte Götter! Unwürdige Götter!}\\
Inu lauschte Saráes Flüstern und seufzte. ``Was ist deine Gabe?'', fragte sie schließlich.\\
``Ich? Hm... ich hab noch keine...''\\
``Also bist du auch ein Lisoe!''\\
Er schüttelte den Kopf. ``Nein. Ich stamme aus einer alten Familie und all meine Ahnen waren 
Chanahe. Also bin ich auch einer. Das kann noch kommen! Das passiert oft erst mit 6 Jahren... meine 
Schwester ist sehr früh.''\\
``Aber du bist älter als 6!'', entschied Inu und grinste ihn feixend an.\\
``Das sagt nichts!'', murrte Semái und wand sich ab. \\
Inu begriff, dass sie ihn gekränkt hatte und es tat ihr schrecklich Leid. ``Ob du ein Chanahe bist 
oder nicht, ich mag dich trotzdem'', sagte sie leise.\\
Er hob den Kopf und streckte ihr die Zunge heraus. ``Dafür, dass du auf jeden Fall eine Lisoe bist, 
mag ich dich auch ein bisschen!'' \\
``Was macht ihr eigentlich hier? Warum baut ihr hier eine Stadt?''\\
``Die Erwachsenen haben sich gestritten. Und da hat mein Vater und zwei andere Familien 
beschlossen, dass sie gehen und eine neue Stadt gründen. Arashad.''\\
``Meine Eltern sind böse darüber'', erzählte Inu.\\
Semái zog die Schultern hoch und nickte. ``Mir gefällt es auch nicht! Zuhause hat es mir besser 
gefallen... nicht so viele Berge. Und es war nicht so heiß. Und es gab viel mehr Wiesen, Seen und 
Wälder! Aber sag mal, du hast doch die Ziege verloren, oder? Auf die solltest du aufpassen?''
Inu nickte und malte sich schon die Tracht Prügel aus, die sie bekommen würde, weil die Ziege weg 
war.\\
``Komm, wir suchen sie, damit du keinen Ärger bekommst!'', bot Semái an und schenkte ihr ein 
Lächeln, das Inu sofort erwidern musste. Als sie seine Hand ergriff, flüsterte Saraé ihr etwas zu, 
aber Inu konnte es nicht verstehen, denn sie lachte laut, als sie sich mit Semái ein Wettrennen 
bot. \\

In den nächsten Tagen hatte Inu lange hin und her überlegt, mit welcher Begründung sie die Eltern 
überreden könnte, wieder in die Stadt zu dürfen. Ihr Vater verließ jeden morgen sehr früh den Hof 
und kam erst am späten Abend von den Baustellen wieder. Daher gab es für Inu und ihre Mutter sehr 
viel zu tun, da sie sich nun zu zweit um den Hof und die Tiere kümmern mussten. Ihre jüngeren 
Geschwister waren dabei nutzlos. Nach einer Woche hatten die Eltern beschlossen, ihre Felder zu 
verpachten, da der Vater mehr Geld auf der Baustelle verdiente, als mit dem Feld und ihre Mutter es 
nicht alleine bewirten konnte. Trotzdem blieben noch die Tiere und der Garten. In diesen Tagen 
sprach Saraé wenig zu ihr. Inu hatte das Gefühl, dass sie nicht wollte, dass sie wieder in die 
Stadt ging und Semái suchte. Aber eines Abends meldete sich das leise Flüstern in ihren Gedanken 
plötzlich wieder.\\
\textbf{Ich merke, er bedeutet dir viel.}\\
\textit{Stört es dich? Ich hatte noch nie einen richtigen Freund.}\\
\textbf{Du hast mich.}\\
\textit{Ja aber... das ist anders. Du bist... mehr. Inu senkte den Kopf. Sie war traurig, dass sie 
es Saraé nicht richtig erklären konnte.}\\
\textbf{In Ordnung. Wenn du morgen deine Arbeit auf dem Hof schnell genug erledigst, darfst du 
vielleicht deinem Vater seine Brotzeit bringen und frei haben.}\\
\textit{Das ist aber alles zu viel... ich war doch gestern auch den ganzen Tag beschäftigt.}\\
\textbf{Jetzt werde ich dir helfen. Du wirst sehen, es wird schneller gehen.}\\
Kurz nach ihrem Gespräch mit Saraé hatte Inu ihren Eltern diesen Vorschlag verkündet. Ihr Vater 
brach in schallendes Gelächter aus und ihre Mutter blickte nur zweifelnd. \\
``Ach lass sie doch'', rief der Vater: ``Wenn sie es schafft, meinetwegen. Solange sie mir nicht 
vor die Füße rennt!''\\

\textbf{Wach auf, wir wollen uns doch beeilen.} \\
Inu schlug die Augen auf und lauschte, wie Saráes Flüstern verklang. Schon stand sie auf den 
Beinen, tänzelte über die schlafenden Geschwister hinweg und stand im Wohnraum. Sie griff sich den 
Besen und fegte schnell den Dreck und das Laub auf dem Hof zusammen. Es war wirklich viel schneller 
als sonst! Der schwere Besen flog regelrecht über den Lehmboden. Kaum hatte Inu die Körner für die 
Hühner gepackt, schon standen diese still in ihrem Gehege. Kein Chaos aus Gackern und 
Flügelschlägen, stattdessen blieben sie artig fern von der Futterkrippe und Inu musste sich nicht 
erst den Weg dorthin frei kämpfen. Bei den Kühen war es ähnlich. Das Mädchen molk die drei Kühe 
schnell, und brachte die Milch und die gesammelten Eier der Hühner in die Wohnstube. Dort stand die 
Mutter und rührte den Frühstücksbrei aus Haferflocken und Wasser an. Sie runzelte die Stirn, als 
sie Inu sah. 
``Wie machst du das? Sonst träumst du immer und brauchst viel länger.``\\
Inu biss sich auf die Lippen und dachte an die letzten Tage. Stimmt, sie hatte oft geträumt. Oder 
besser gesagt, sich während der Arbeit mit Saraé unterhalten und ihren Geschichten gelauscht. Heute 
waren nur knappe Anweisungen von Saraé gekommen, oder Fragen, was als nächstes zu tun sei.\\
``Die Tiere sind gefüttert, getränkt und gemolken?''\\
Inu nickte. ``Aber das ausmisten ist zu schwer für mich'', murmelte sie betreten. \\
Ein weiteres Nicken ihrer Mutter. ``Dann lass uns jetzt erst einmal essen. Und dann machst du noch 
den Garten und bringst das Pferd zum Hof der Kassar. Sie wollten es sich heute für das Feld 
ausleihen. Wie das Geschirr dran gehört, weißt du ja. Und vorher gut das Fell putzen, sonst 
scheuert es. Gib ihm eine extra Portion Hafer und am Abend holst du es wieder ab und bringst es in 
den Stall.''\\
Inu versprach daran zu denken und setzte sich neben ihren Geschwistern an den Tisch. Das Abspülen 
nach dem Essen und das Holen des Feuerholzes waren die Aufgaben der Jüngeren. Inu entspannte sich 
beim Essen und beobachtete ihre Mutter, wie sie die Kleinste mit dem Holzlöffel fütterte. 
Es war bereits später Vormittag, als Inu in den Gemüsegarten hinter dem Wohnhaus trat und dort mit 
Staunen sah, dass das Unkraut zwischen dem nützlichen Pflanzen braun und welk geworden war. \\
\textit{Wie machst du das ?, fragte Inu wieder, während sie das tote Unkraut mühelos aus der Erde 
entfernte. }\\
\textbf{Die Pflanzen und Tiere gehorchen einer Göttin.}\\
``Göttin?'', rief Inu laut und hielt in der Bewegung inne: ``So wie Kanto und Anta?''\\
\textbf{So ähnlich.}\\
``Ich wusste nicht, dass es noch mehr Götter als Sonne und Mond gibt...''\\
\textbf{Du weißt vieles nicht. Du weißt nicht einmal, für was ihr sie ehrt. Ja, sie haben unter 
anderem die Welt, die du kennst, oder glaubst zu kennen, erschaffen.}\\
Inu entging nicht der leise Zorn in Saráes Stimme.\\
\textit{Du magst sie nicht? Warum?}\\
Dazu kam erst einmal nur Schweigen. \\
\textbf{Ach Inu... lerne erst einmal etwas über Kanto und Anta, sehe deine Welt und gelange an 
deren Grenzen. Dann darfst du mir diese Frage noch einmal stellen, wenn es denn dann noch notwendig 
sein wird.}\\
Inu ließ diese Worte erst einmal so stehen. Saraé hatte ja recht. Sie wusste nicht viel von Kanto 
und Anta. Nur, dass sie die Welt erschaffen haben und als Sonne und Mond darüber wachten. Ihre 
Familie hatte sich nie groß mit den Göttern beschäftigt, aber ihr kam der Gedanke, dass Semái 
vielleicht mehr wusste. Immerhin war er ein Chanahe... ein begnadeter von Kanto und Anta. 
Vielleicht konnten die Chanahe sogar mit Kanto und Anta sprechen, so wie sie selbst mit Saraé!\\
\textit{Bist du müde?}, fragte Inu Saraé und tastete nach dem Amulett. Die Göttin kam ihr auf 
einmal so weit weg vor. \\
\textbf{Etwas... Es ist anstrengend dir zu helfen. Ohne einen Körper, verbraucht es viel Kraft. 
Außerdem war ich lange erstarrt und meine Kraft muss erst wieder wachsen. Aber du bist ja bald 
fertig und wenn du mit Semái unterwegs bist, werde ich mich ausruhen. Wenn du Hilfe brauchst, rufe 
mich. Ich werde jeden deiner Rufe hören, egal wo ich bin oder wie tief ich schlafe.}\\
Ihre Worte beruhigten Inu und sie machte sich schnell auf zum Stall, um das Pferd zu putzen und das 
Zuggeschirr anzulegen. Als sie mit all ihren Arbeiten fertig war, hatte die Sonne ihren Zenit 
bereits überschritten. Inu eilte schnell zur Mutter und nahm den gefüllten Korb mit. Darin konnte 
Inu weiches Brot, eingewickelten Käse und eine Lederflasche erkennen. Sie verabschiedete sich und 
machte sich auf den Weg zu den Baustellen. Es ist seltsam, dachte Inu: Obwohl die Erwachsenen alle 
über die Chanahe schimpfen, arbeiten die meisten Männer jetzt auf der Baustelle. Und wir haben viel 
mehr leckeres Essen als früher! Und Mama hat neuen Stoff, aus dem sie uns Kleider nähen will.
Zu Inus Überraschung schwieg Saraé. Sie schien wirklich zu schlafen. \\

``Danke'', brummte der Vater und zerzauste Inus Haar. Eine fast schon liebevolle Geste, die das 
Kindergesicht zum strahlen brachte. \\
``Darf ich jetzt spielen gehen?''\\
``Steh niemandem im Weg herum. Und stell nichts an!''\\
Sie nickte hastig und rannte schon durch die zukünftigen Straßen von Arashad. Nach Semái zu rufen, 
traute sie sich nicht, daher spitzte sie in jede Lücke und auf jede Baustelle. Sie huschte zwischen 
den arbeitenden Männern hindurch und an anderen Kindern, vermutlich alle Chanahe, vorbei. \\
``Hey, du Wirbelwind!'', rief ihr Jemand zu und Inu bremste ihren Lauf abrupt ab. \\
Sie folgte der Stimme und blickte an dem bereits fertiggestellten Haus empor. Die Wand war weiß 
verputzt, Fenster und Türen aus einem dunklen Holz. Das Dach war mit dunklen Ziegeln bedeckt. Und 
in den Fenstern war tatsächlich Glas. In einem geöffneten Fenster entdeckte sie Semáis zerzausten 
Haarschopf. \\
``Kannst du runter kommen?'', rief sie ihm zu. \\
Er nickte und war aus dem Fenster verschwunden. Wenige Minuten später stand er schon neben ihr. \\
``Das Haus ist riesig!''\\
Er zuckte mit den Schultern. ``Naja... es geht. Was hast du vor?''\\
``Soll ich dir die Gegend zeigen? Den Wald?'', schlug Inu begeistert vor und Semái lächelte 
zustimmend. 
Sie rannten über die Wiesen und an Inus Ansiedlung vorbei. Stundenlang jagten sie sich durch den 
lichten Wald und versteckten sich voreinander. Am späten Nachmittag kamen sie, völlig außer Atem, 
zu dem kleinen Waldsee, der von Beginn an Inus eigentliches Ziel gewesen war. Der See lag nicht 
weit von der Stelle, an der sie Saráes Amulett gefunden hatte. Keuchend und nach Luft ringend 
ließen sie sich auf den moosbedeckten Boden fallen und Inu betrachtete erfreut Semáis faszinierten 
Gesichtsausdruck. Sie war stolz, dass sie dem neuen Freund einen so tollen Ort zeigen konnte. 
Der kühle Frühlingswind zog leise raschelnd durch die Baumkronen. Der See lag still und friedlich 
da, während sich die Äste zweier Bäume weit über ihr neigten und deren Blätter die Wasseroberfläche 
fast berührten. Hinter dem See ragte eine Felswand, bewachsen mit allerlei Plfanzen und vor allem 
Efeu, auf. Dünne Wasserrinnsale zogen sich an der Wand hinunter. Inu folgte der Felswand hinauf und 
betrachtete den Wald, der dort oben weiter ging. Sie war noch nie den Hang hinaufgeklettert. Hatte 
es sich bisher nie getraut. Jetzt wurde sie neugierig. Inu warf einen Blick zu Semái, dessen Augen 
fest auf dem grünen Waldsee gerichtet waren. 
Ihr wurde plötzlich mulmig zumute und sie tastete nach Saráes Amulett. \\
``Sag mal...'', murmelte Inu, während sich das Amulett in ihren Fingern drehte: ``Was weißt du über 
Götter?''\\
Semái blickte sie nachdenklich an und zuckte mit den Schultern. \\
``Was meinst du? Ich kenne die Geschichten über Kanto und Anta.''\\
Inu nickte. ``Erzähl mir eine.''\\
``Kennst du die nicht selbst?  Oder erzählt ihr Lisoe euch nicht so viel darüber?''\\
Inu nickte nur schüchtern und zog die Beine an. \\
``In Ordnung... hm... meine Großmutter erzählte immer, es gab einen Ort, dort lebten nur Götter! 
Unzählige Götter. Kanto und Anta lebten dort auch. Aber eines Tages begannen sie sich zu streiten. 
Es herrschte ein furchtbarer Krieg, da keiner nachgeben wollte.''\\
``Götter streiten?'', rief Inu überrascht: ``Ich dachte, so was machen die gar nicht!''\\
Semái grinste. ``Kanto und Anta haben es auch nicht gemacht. Sie wollten den Streit schlichten. 
Aber irgendwann sahen sie ein, dass den anderen Göttern nicht zu helfen war und sie gingen. Sie 
verließen ihre Heimat und erschufen eine eigene Welt. Unsere.''\\
\textbf{Sie verließen das Heim unter vielen Verwünschungen und Flüchen}, wisperte Saraé: 
\textbf{Sie verließen ihre Familie und Gefährten, ließen sie zurück... warfen niemals einen Blick 
hinter sich. Sie gingen, weil sie den Kampf nicht gewinnen konnten. Sie wollten ebenso ihre Ziele 
durchsetzten wie alle anderen von uns. Manche traten für Gerechtigkeit ein, andere für komplette 
Gleichheit, wiederum andere für Glück oder Aufopferung. Kanto trat für die Freiheit und Chaos ein, 
Anta für Friede und Ordnung. All diese Werte können aber ohne ihre Schattenseiten nicht existieren 
und keiner von uns konnte die Schattenseiten der Anderen akzeptieren. Deshalb beschlossen sie diese 
Welten zu verbinden. Die eine ist der Schatten der anderen. Die andere der Schatten der einen.}\\
``Sie schufen eine Welt für die Freiheit'', erklärte Semái: ``Und eine Welt des Friedens.''\\
``Und welche ist unsere?''\\
``Überlege doch mal!''\\
Inu dachte laut nach. ``Bei uns gibt es oft Streit und Kampf zwischen Familien und Dörfern. Also 
kein Friede.''\\
``Wir sind die Freien?'', fragte Inu mit Stolz in der Stimme.\\
``Vielleicht'', entgegnete Semái grüblerisch: ``Wenn man frei ist, darf man alles machen, ohne auf 
andere Rücksicht zu nehmen, oder? Das würde erklären, warum sich so viele Familien streiten und 
Dörfer sich bekämpfen. Weil sie nur an ihre Freiheit denken.''\\\\
``Aber wir sind doch nicht im Chaos... wir haben Ordnung'', entschied Inu.
``Vielleicht etwas... wer weiß. Aber die Geschichte ist noch nicht zu ende, hör zu! Sie erschufen 
eine Welt der Freiheit und des Chaos und eine Welt des Friedens und der Ordnung. Und obwohl sie 
diese Welten gemeinsam erschufen und behüten, wollten sie verhindern, dass die Wesen aus der einen 
jemals zur anderen gelangen. Aber für einander mussten sie trotzdem Toren als Verbindung 
erschaffen. Also schufen sie eine dritte, viel kleinere Welt. Sie liegt zwischen Freiheit und 
Frieden, zwischen Chaos und Ordnung. Die Hölle.''\\
\textbf{Das Imuril}\\
``Ein Ort, den kein Mensch je betreten kann und selbst wenn, keinen Tag überleben wird. Die Hölle 
ist verglühend heiß und brennend kalt gleichzeitig! Voller giftigen Früchten und grässlichen 
Raubtieren! Und es gibt einen Wächter, erzählte meine Großmutter. Aber viel wusste sie von der 
Hölle auch nicht.''\\
``Einen Höllenwächter?''\\
\textbf{Die Lio.}\\
``Das Imuril'', flüsterte Inu nach: ``Die Lio...''\\
``Hast du was gesagt?''\\
Das Mädchen schüttelte heftig den Kopf. ``Nein nein... sag mal, glaubst du daran, dass es noch mehr 
Götter bei uns gibt?''\\
Er zuckte mit den Schultern. ``Das ist Kanto und Antas Welt. Wenn es noch mehr hier geben würde, 
wären sie bestimmt nicht begeistert. Aber andererseits... es sind Götter. Wer weiß, wofür die sich 
interessieren. Aber ich muss jetzt gehen, sonst bekomme ich noch Ärger von meiner Mutter...''\\
``Kennst du noch mehr Geschichten?'', platzte es aus Inu heraus: ``Wirst du mir noch mehr über 
Kanto und Anta erzählen?''\\
Er grinste und strich sich über das dunkle Haar. ``Vielleicht. Kommst du morgen wieder in die 
Stadt?''\\
``Ich versuch's!''\\
``Dann bis bald Wirbelwind!'', lachte Semái und winkte ihr zu Abschied.\\
``Wenn du diese Geschichten auch kennst, warum hast du sie mir nie erzählt?'', fragte Inu leise 
Saraé.\\
\textbf{Mich interessiert es, was ihr Menschen euch über Kanto und Anta erzählt. Und ich kenne die 
Geschichten nicht. Ich kenne sie etwas anders. Und ich kenne die Namen, die ihr längst vergessen 
habt. Aber weißt du was? Der Junge hat etwas wichtiges angesprochen. Wer weiß, wofür sich die 
Götter interessieren?}\\
Leises Kichern. Inu runzelte nachdenklich die Stirn und machte sich auch auf den Heimweg. \\

Sie traf sich beinahe jeden Tag mit Semái. Meistens schlichen die Kinder durch die Natur, 
erkundeten Wälder und Wiesen. Der Junge erzählte von Kanto und Anta, von Geistern und den Ahnen. 
Inu gab Saráes fremde Geschichten wieder und gemeinsam phantasierten sie über das Ende der Welt. 
Was dort wohl auf sie warten würde? Gab es überhaupt eines? Die Kinder malten sich unzählige 
Möglichkeiten und Ereignisse aus. \\
An diesem Abend saßen sie wieder am Geröllhang. Es war schon spät, aber keiner der Beiden machte 
Anstalten, aufzustehen und sich zu verabschieden. Inu beobachtete Semái lange und kam zu dem 
Schluss, dass ihr Freund bedrückt war. Hin und her gerissen überlegte sie, ob sie ihn fragen 
sollte. Aber sie wagte es nicht. Sie hatten so viele Tage miteinander verbracht, so viele 
Geschichten erzählt und so viele Orte entdeckt, aber bis auf ihre Namen, nichts übereinander 
ausgetauscht. Außerdem war Inu nervös. Der Geröllhang war ihr viel zu nahe bei Arashad. Und sie 
wusste ja, dass sie dort unwillkommen war. \\
``Du bist unglücklich?'', fragte sie schließlich.\\
Semái legte den Kopf schief und starrte zum Fluss hinunter. ``Es ist manchmal alles so beschissen.''
Er sprach so leise, dass Inu ihn fast nicht verstand. Sie musste sich schon zu ihm hin beugen, um 
seine Worte zu hören. \\
``Was kann für einen Chanahe schlimm sein?'', fragte sie ohne groß darüber nachzudenken: ``Ihr habt 
ein riesiges Haus, genug Sachen zum eintauschen, müsst nicht bei jedem Wetter arbeiten und seid von 
den Göttern begnadet.''\\
``Meinst du das ernst?'', rief er aus und stand auf: ``Wie kommst so auf so was? Das stimmt nicht! 
Den Besitzt haben sich meine Eltern erarbeitet! Und die Gabe... die hab ich ja gar nicht.''\\
Bei den letzten Worten brach seine Stimme und Inu sah die Tränen in seinen Augen. \\
``Ich bin schon acht und habe keine Gabe. Weißt du, was das heißt? Dann kann ich gehen! Gabenlose 
sind für die Familie eine Schande!''\\
Er ließ den Kopf hängen und seine Schultern zitterten. Zögernd legte sie ihm eine Hand auf die 
Schulter. Lange überlegte sie, was sie sagen sollte. Erwachsene hätten jetzt bestimmt die richtigen 
Worte gefunden. Oder sie hätten einfach etwas sinnloses geredet, etwas, dass gar nicht half. Inu 
beschloss zu schweigen. 
Die Sonne war ein ganzes Stück am Himmel entlang gewandert, als Inu das nächste Mal aufblickte.\\
``Hey! Was macht denn diese dreckige Lisoe schon wieder hier?!'' \\
Vor Schreck sprang Inu schnell auf die Beine. Sie sah eine Gruppe Jungen und Mädchen auf den 
Geröllhang zukommen und bekam es mit der Angst zu tun. Ja, Saraé war bei ihr, und Semái. Aber würde 
das helfen? Es waren so viele! In der Angst schaffte Inu es nicht einmal die Kinder zu zählen. Am 
liebsten wäre sie weg gerannt, aber sie fürchtete, dass die Meute ihr zu ihrem Heim folgen würde. \\
``Saraé?'', flehte sie leise und ihre Finger umschlangen das zierliche Amulett. Schweigen war die 
einzige Antwort, die sie erhielt. Semái hatte sich der Gruppe zugewandt und schnell die Tränen fort 
gewischt. \\
``Hast du ein Problem Meroka?''\\
Meroka Horosan , der wieder an der Spitze der Meute stand und wohl ihr Anführer war, stemmte die 
Hände in die Hüften. ``Hau ab, Malaza! Und schäm dich, dass du dich mit dieser Lisoe 
herumtreibst!''\\
``Ach, soll er doch! Er ist doch selbst einer!'', höhnte ein Mädchen. \\
``Selbst seine Schwester hat die Gabe! Und die ist zwei'', rief ein pummeliger Junge und zeigte 
lachend auf Semái.\\
Inus Wangen färbten rot vor Zorn und sie spürte eine Hitze in sich aufkommen. Ihre Angst wurde 
bedeckt von Zorn und Hass gegenüber den Chanahe Kindern. Wie konnten sie nur Semái, der immer lieb 
und nett war und immer so tolle Geschichten erzählte, so behandeln? Semái war aufgestanden. So 
ernst hatte sie ihn noch nie gesagt.\\
``Prügeln wir zurück?'', fragte Inu, die keine Sekunde zweifelte, dass er auf ihrer Seite stand.
Doch er schüttelte den Kopf und trat einen Schritt vor.\\
``Was willst du eigentlich Meroka? Geh doch mit deinen Hündchen woanders spielen. Soll ich es 
meinen Eltern erzählen? Inu hat jedes Recht hier zu sein. Ihr Vater baut dein Haus auf. Vielleicht 
solltest du lieber dankbar sein. Wenn Inus Vater nicht die Steine schleppen würde, müsstest du es 
nämlich tun. Und das weißt du genau.''\\
``Ich habe nichts gegen dich, Malaza'', entschied der junge Anführer und verschränkte die Arme vor 
der Brust. Er warf einen Blick hinter sich. ``Nur die Lisoe soll verschwinden... Malaza ist ein 
Chanahe wie wir. Er wohnt hier und darf bleiben.''\\
``Die Lisoe soll verschwinden!'', schrie der dicke Junge und schleuderte etwas durch die Luft. Inu 
erkannte einen bunt bestickten Lederbeutel, in dem die Chanahe Kinder ihre Spielsteine 
aufbewahrten. Sie kniff die Augen zusammen, als der gefüllte Beutel auf sie zuflog. Im Selben 
Moment hörte sie, wie der Gegenstand Semái traf. Semái, der viel näher am Abgrund stand als sie. 
Inu streckte ihre Hand aus, aber zu langsam. Der Junge wurde an der Schulter getroffen und verlor 
das Gleichgewicht. Einzelne Steine kullerten den Hang hinab und keine Sekunde später  verlor Semái 
den Halt. Inu stieß einen Schrei aus und auch einige der Chanahe Kinder erbleichten. Meroka machte 
sogar einen Satz nach vorne.  Das Geröll prasselte den Hang hinunter. Steine, größer als ihr Kopf, 
hatten sich gelöst.\\
\textit{Nein}, dachte Inu.\\
Und während die anderen Kinder schrien und fort rannten, stand sie einfach nur da und sah hinab. Es 
war, als sähe sie von der Luft aus auf sich selbst. Das kleine Mädchen stieg in den Himmel und 
blickte starr auf die Szenerie. Sie sah wie der Junge sich überschlug. Er rollte quer hinab, 
umgeben von Kiesel und Steinen. Und während sie, seine Freundin, da stand, rannte der andere Junge 
ebenfalls los. Aber nicht den anderen Kindern hinterher, sondern auf den Hang zu. Inu sah, wie sich 
sein Mund zu einem Ruf öffnete, wie er die Hand ausstreckte und ins Leere griff. Und sie? Sie stand 
da, sah ihm zu und blinzelte gegen den Staub an. \\
``Semái!'' Merokas Ruf hallte durch die Stille die eintrat, als Semái am Ufer des Flusses zum 
liegen kam. Nur das leise Prasseln weiterer Kiesel untermalte das Geschehen. Inu beobachtete mit 
einer sachlichen Neugierde, die sie selbst erschreckte, wie Meroka leichtfüßig den Hang hinunter 
sprang. Als würde er fliegen. Aber die winzigen Steinchen die seine Füße lockerten zeugten davon, 
dass sie doch den Boden berührten. Währenddessen breitete sich langsam eine dunkle Blutlache aus. \\
``Was machst du mit mir?'', flüsterte Inu leise.\\
Saraé schwieg, aber das Mädchen spürte, wie die Göttin sich abwandte und ihr die Kontrolle über 
ihre Glieder zurückgab. Endlich setzte Inu sich in Bewegung und lief einige Schritte den Hang 
entlang um eine flachere Stelle zu finden. Fast schon krabbelnd kroch sie hinunter um dann die 
letzten Meter am Fluss entlang zu sprinten. Meroka hatte Tränen in den Augen, als sie bei den 
beiden Jungen ankam. Der sonst so vorlaute Chanahe blickte zu ihr auf und ein Schluchzen kam über 
seine Lippen.\\
\textbf{Und so schnell fallen die Begnadeten.}\\
``Es tut mir leid!'', schrie er, als könnte er alle Schuld abwenden, wenn er es nur laut genug 
sagte: ``Das wollte ich nicht!''\\
Inu ließ sich neben ihm auf die Knie nieder und spürte wie die spitzen Steine sich in ihre Haut 
bohrten. Es war das erste Mal, dass Inu so viel Blut eines Menschen sah. Ratlos musterte sie jede 
Einzelheit. Semái lag auf der Seite, die Arme noch in einer Geste erhoben, als wollte er weiterhin 
seinen Kopf schützen. Eines seiner Beine war unnatürlich verdreht. Und das Blut breitete sich 
weiter aus.\\
``Woher kommt das Blut?'', fragte Inu leise, da sie keine Wunde erkennen konnte. Die Verletzung 
musste auf der Seite sein, auf der er lag. Während Meroka zitternd über Semái kauerte und nur immer 
wieder den Kopf schüttelte, saß Inu wieder stumm da. Ihre Fingernägel bohrten sich in die 
Handballen. \\
``Meinst du, die anderen holen einen Erwachsenen?'', fragte sie dann. Die Worte kamen ihr nur 
mühsam über die Lippen.\\
Meroka sah sie an und zum ersten Mal blickte sie dem älteren Jungen länger in die Augen. Sie waren 
blau. Und sie sahen gar nicht so böse aus, wie sie gedacht hatte. Zumindest jetzt nicht, während er 
weinte und zitterte. Er schien nach Worten zu suchen. Sein Mund öffnete sich und schloss sich 
gleich darauf wieder. Dann flüsterte er leise. ``Ich weiß nicht, ob er wieder aufwachen wird.''\\
``Du meinst, Semái stirbt?''\\
``Vielleicht ist er schon tot…''\\
Das Mädchen stieß einen Schrei aus, den kein lebendes Wesen hören konnte. Nichts war von außen 
sichtbar über dieses Gefühl, dass einer laut schreienden Angst gleich kam. Nur Saraé hörte es. Und 
sie war da. \\
\textbf{Er blutet.}\\
\textit{Das sehe ich!}, dachte Inu.\\
\textbf{Nein. Nicht nur am Kopf. Auch innen.}\\
Das Mädchen verstand den Sinn der Worte nicht ganz, aber sie konnte nun nicht mehr an sich halten 
und griff nach den Jungen. So oft hatte sie ihn in den letzten Tagen berührt. Er hatte sie sogar 
ein paar mal auf seine Schulter gehoben, wenn sie Obst von den Bäumen holen wollten. Aber jetzt 
wirkte sein schlaffer Körper so viel schwerer, als sie es sich vorgestellt hatte. Inu zerrte an 
ihn, bis sie Semáis Oberkörper in den Armen hielt. Erst jetzt kamen die Tränen, die das kleine 
Mädchen die ganze Zeit zurückgehalten hatte.\\
``Er darf nicht sterben!'', murmelte sie. \textit{Er ist doch der einzige Mensch der mich mag!}\\
Meroka rappelte sich hastig auf und sah sie zögernd an. ``Ich glaube nicht, dass es gut war, dass 
du ihn bewegt hast… mein Vater sagt immer, wenn jemand fällt und er alleine nicht wieder aufsteht, 
sollen wir ihn liegen lassen und einen Erwachsenen holen.''\\
``Fallt ihr so oft?'', knurrte Inu ihn böse an.\\
Er zuckte mit den Schultern. ``Ja. Aber nicht so schlimm. Chanahe stürzen nicht schlimm...''\\
Trotz lag in ihrem Blick, als sie zu ihm empor starrte. \\
``Dann ist Semái vielleicht wirklich kein Chanahe.''\\
Meroka holte tief Luft. ``Ich geh meinen Vater holen.'' \\
Hastig wischte der Junge sich die Tränen ab und machte sich auf den Weg den Hang empor. Inu sah ihm 
nicht noch einmal zu, aber lauschte auf seine geschickten Bewegungen. \\
``Hätte Semái seine Gabe, wäre er nicht gestürzt'', flüsterte sie: ``Wieso haben Kanto und Anta 
ihn ausgelassen?''\\
\textbf{Weil sie ungerecht sind. Vielleicht wollten sie nie, dass er lebt. Dass er erwachsen 
wird.}\\
Inu‘s Finger gruben sich in seine Haut und sie drückte ihn fest an sich. ``Aber wieso?''\\
\textbf{Vielleicht… haben sie Angst vor dem, was Semái tun wird, wenn er erwachsen sein wird.}\\
``Götter brauchen sich nicht zu fürchten.''\\
Saraés Stimme klang traurig, als sie antwortete. \textbf{Mein Mädchen im Mondschein… alles, was wir 
Götter tun, geschieht aus einem einzigen Grund. Weil wir Angst haben. Angst, dass wir vergessen 
werden. Angst, dass wir verdrängt werden. Angst, dass wir verlöschen.}\\
``Dann tu etwas. Rette ihn!''\\
\textbf{Mädchen im Mondschein…}\\
``Du hasst sie doch!'', schrie Inu: ``Du hasst Kanto und Anta! Wenn sie Semái sterben lassen 
wollen, dann musst du ihn retten!''\\
\textbf{Das kann ich nicht.}\\
``Du lügst.''\\
Einen Moment lang hörte Inu nur ihren eigenen, keuchenden Atem. Dann erklang Saráes Stimme als ein 
leises Wispern, getragen vom Wind. \textbf{Dann gib du mir die Kraft. Schwöre mir dein Leben. 
Schwöre mir dein Blut. Schwöre mir all deine Gedanken und Wünsche, deinen Glauben und dein Herz. 
Und deine Seele.}\\
Inu erstarrte. ``Ohne meine Seele werde ich nicht wiedergeboren'', wiederholte sie leise die Worte 
ihrer Mutter. Schon immer hatte man ihr erzählt, dass man seine Seele niemals an böse Geister und 
Dämonen geben darf, sonst stirbt man für immer.\\
\textbf{Du wirst nicht wiedergeboren, weil du nicht sterben wirst, Mädchen im Mondschein.}\\
Sie biss sich auf die Zunge und sah hinab auf ihren Freund. \\
\textbf{Ich habe alleine nicht die Kraft, Inu. Du hast gefleht, dass ich ihn rette, dann gib mir 
die Kraft. Alleine kann ich es nicht.}\\
Ihre Stimme klang so ruhig. So sachlich. Nicht wie böse Geister, die nach ihrer Seele gierten. 
Saráe klang eher gelangweilt. Ihr war es egal ob der Junge starb oder gerettet wurde.\\
``Ich schwöre dir mein Blut, mein Leben und meine Seele. Ich schwöre dir meine Gedanken, meine 
Wünsche und meinen Glauben'', wiederholte sie flüsternd, während sie beobachtete, wie das Blut 
weiter aus einer Wunde an seinem Kopf floss und ihre Kleidung und Haut benetzte.\\
Saráe sah in das Kind hinein und betrachtete die kleine Seele, die da am Fluss kniete. Der 
körperlosen Göttin entging es nicht, dass das Herz ihr nicht geschworen wurde. Aber sie dachte 
nicht lange darüber nach, denn sie konzentrierte sich auf das Herz des Jungen, den Inu in den Armen 
hielt. Sie konzentrierte sich auf das Blut, welches seine gewohnten Bahnen verlassen hatte. Saraé 
drängte das Blut der inneren Verletzungen zurück und schloss diese unsichtbaren Wunden. 
Währenddessen schloss Inu die Augen und sank kraftlos in die Dunkelheit. \\

Als das Mädchen die Augen öffnete, sah es direkt in die dunklen Augen eine Frau. Sie sah schön aus. 
Sie war die schönste Frau, die sie bisher gesehen hatte. Ihre Haut war glatt und blass, das Haar 
schwarz, die Wimpern dunkel und geschwungen. Auch ihre Lippen hätten schön ausgesehen, wenn die 
Frau sie nicht zusammengekniffen hätte. Inu vermutete gleich, dass es eine Chanahe war. Die Lisoe 
sahen nie so hübsch aus, weil sie so viel arbeiten mussten. Sie waren einfach dreckiger und müder. 
Die Chanahe lehnte sich zurück und betrachtete das Mädchen mit einem kalten Blick. Inu biss sich 
auf die Zunge vor Angst und wagte nicht etwas zu sagen. Aber die Frau schien auch nicht das 
Verlangen zu haben, mit ihr zu reden. \\
``Wo bin ich?'', flüsterte Inu schließlich.\\
Die dunklen Augen wurden schmaler.\\
``In Arashad.''\\
Einen langen Moment lang starrten sie einander an, bis die Frau schließlich den Blick abwandte und 
sich erhob. Ihre Bewegungen waren elegant und geschmeidig, wie sie nur von einer starken Chanahe 
sein konnten. Sie schenkte sich aus einen Krug ein Glas Wasser ein, eher sie zu Inu sprach: ``Du 
bist also der Grund, warum mein Sohn ständig fort läuft.''\\
Inu brauchte einen kurzen Moment um zu verstehen. ``Du bist Semáis Mutter?''\\
Sie erinnerte sich dran, dass er erzählte, wie schnell seine Mutter rennen konnte und Inu 
betrachtete die Frau nun mit großen Augen. \\
``Eine kleine, dreckige Lisoe'', fuhr sie fort.\\
Trotzig hob Inu das Kinn und kletterte von der gepolsterten Bank. ``Er ist nicht begnadet!''\\
Sie war wirklich schnell. So schnell, dass Inu es kaum sehen konnte, wie die Frau sich in Bewegung 
setzte. Aber der Schmerz, der war sofort da. Der Schlag schubste sie zurück auf die Bank und 
trieb ihr Tränen in die Augen. \\
``Miststück! Lisoe! Wage es nicht noch einmal, meine Familie zu beleidigen.''\\
``Wo ist Semái? Ist er wieder gesund?''\\
Die Frau richtete sich auf und faltete die Hände. Die Aggression verschwand wieder hinter ihrem 
kaltem Gesichtsausdruck. \\
``Er schläft, aber sein Herz schlägt stark. Und du wirst jetzt gehen und nie wieder in seine Nähe 
kommen. Er ist ein Chanahe und gehört einer alten, starken Familie an. Semái sollte keinen Kontakt 
mit Lisoe haben.''\\
``Aber er mag mich!''\\
``Hast du schon einmal darüber nachgedacht, dass er vielleicht von den Göttern nicht begnadet wird, 
weil er mit dir Zeit verbringt?''\\
Es war Saraé, die wortlos die Führung übernahm. Sie war es, die das Mädchen aufstehen und ohne 
einen Blick zurück zu werfen, den Raum verlassen ließ. Inu in ihr drinnen dagegen tobte und schrie, 
weinte und schlug um sich. Wie gerne hätte sie dieser Frau weh getan! Aber das ließ Saraé nicht zu 
und Inu fehlte die Möglichkeit, die Göttin wie bisher aus ihren Körper zu verbannen. \\
Erst, als Inu‘s Füße sie aus der Stadt hinaus trugen, über Wiesen und Feldwege, sprach Saraé zu 
ihr.\\
\textbf{Du darfst dich nicht so benehmen, Inu.}\\

Selbst wenn Saraé es nicht verboten hätte, machte Inu in den nächsten Wochen keine Anstalten noch 
einmal nach Arashad zu gehen. Das Mädchen ging ihren täglichen Arbeiten nach, versorgte die Tiere 
und den kleinen Gemüsegarten. Auch in den Wald verschwand sie nicht mehr sondern verzog sich in das 
Haus und versuchte sich an Stickarbeiten, die sie vielleicht verkaufen könnten. Ihre Eltern entging 
dieser Wandel nicht, aber sie hatten andere Sorgen als sich nach der Ursache zu fragen. Solange Inu 
ihre Aufgaben erledigte, taten sie so, als hätte sich nichts verändert. Saraé versuchte es immer 
wieder, aber Inu reagierte kaum noch auf ihre Gedanken. Das Mädchen war so wütend auf Saraé. Nein, 
sie war wütend auf alles. \\
Das alles änderte sich erst wieder, als es eines abends noch an der Tür klopfte. Es war der Vater 
der – nachdem er mit seiner Frau einen misstrauischen Blick ausgetauscht hatte – sich erhob und 
öffnete. Neugierde erfasste die Kinder und Inu erhob sich eilig wieder von ihrem Schlaflager um an 
der großen Gestalt ihres Vaters vorbei schauen zu können.\\
\textbf{Na… endlich wieder glücklich?}, fragte Saraé resignierend.\\
Und wirklich konnte Inu sich ein Grinsen nicht verkneifen. Semái stand seinem Vater gegebenüber. 
Trotz den deutlichen Größenunterschied wirkte der Junge alles andere als eingeschüchtert, während 
er auf blickte und den Augenkontakt mit dem Lisoe suchte. \\
``Ich will Inu sehen'', sagte Semái in seinem ihr so bekannten ersten Tonfall. \\
``So?'', entgegnete der Vater und runzelte die Stirn: ``Wer bist du überhaupt?''\\
Es war keinem Mitglied der Familie entgangen, dass der Junge gute Kleidung trug und einen Akzent 
hatte. Semái schwieg, wirkte dabei aber nicht nervös oder unsicher. Stattdessen sah er an Inus 
Vater vorbei und begegnete ihren Blick.\\
``Ich komme nacher wieder'', erklärte Inu nur und schlüpfte an ihrem ratlosen Vater hinaus auf den 
Hof. Sie ergriff Semáis Hand und zog ihn fort, ehe der Vater sich doch noch entscheiden würde, wie 
er mit dieser unerwarteten Situation umgehen sollte. \\
Eine Weile liefen die Kinder schweigend nebeneinander her, während sie einander weiterhin an der 
Hand hielten. Die Sonne versank langsam. Einzelne Lichtstrahle blitzten über den dämmrigen Himmel 
und blendeten Inu flüchtig.\\
``Wir sollten nicht in den Wald gehen'', sagte sie schließlich: ``Ich war noch nie nachts im 
Wald.''\\
\textbf{Mein Mädchen im Mondschein, jetzt lügst du aber}, spottete Saraé.\\
\textit{Da war ich ja nicht alleine. Du warst da.}\\
\textbf{Ich bin immer bei dir.}\\
``Ich schon'', antwortete Semái: ``Da ist nichts schlimmes.''\\
``Doch! Böse Tiere!''\\
Der Junge zuckte mit den Schultern. ``Ich habe schon oft welche gesehen. Sie haben freundlich 
gegrüßt und sind vorbei gezogen. Die meisten wollen zu dieser Zeit nach hause zu ihrer Familie oder 
sind gerade auf den Weg zum arbeiten und Essen besorgen.''\\
Inu funkelte ihn wütend an. ``Hör auf mit deinen dämlichen Geschichten!''\\
``Das ist aber eine wahre Geschichte'', verteidigte Semái sich. ``Hey! Alle meine Geschichten sind 
wahr.''\\
Inu schüttelte nur entrüstet den Kopf, traute sich aber trotz Saraés und Semáis Versprechen nicht, 
seine Hand los zu lassen. \\
``Mir tun Tiere nichts'', sagte der Junge versöhnlich: ``Keine Ahnung wieso. Die Hunde hören auf zu 
kläffen, wenn sie mich sehen. Die Raubkatzen schauen mich nur an und gehen dann weg. Und letztens 
habe ich ein wildes Pferd gestreichelt!''\\
``Die kann man nicht streicheln! Die rennen immer weg.''\\
``Ja, weil sie Angst haben vor den Menschen. Sie wollen halt nicht das Abendessen sein. Kann man 
doch verstehen.''\\
``Du warst also schon im Wald, bevor du zu mir gekommen bist?'', rief Inu.\\
Semái nickte. ``Ja… aber da war es noch später in der Nacht. Ich bin weggelaufen. Meine Mutter 
lässt mich nicht alleine aus den Haus. Sogar in der Stadt sind immer meine Eltern dabei'', erklärte 
er frustriert: ``Es ist so langweilig.''\\
``Deine Mutter ist bescheuert'', murmelte Inu. ``Was machst du denn dann den ganzen Tag lang?''\\
``Lernen'', seufzte er: ``Also, ich war auch lange krank wegen dem Sturz. Ich durfte lange nicht 
aufstehen. Aber selbst dann musste ich lernen.''\\
``So viel gibt es doch gar nicht, was man lernen kann'', entschied Inu: ``Deine Mutter ist einfach 
gemein.''\\
``Doch. Es gibt ganz viele Sprachen. Fast jede Siedlung hat eine andere Sprache oder zumindest 
einen eigenen Dialekt. Das weißt du nur nicht, weil du noch nie aus deiner rausgekommen bist.''\\
Inu verzog das Gesicht und schaute finster, sagte aber nichts. \textit{Er findet mich dumm.}
\textbf{Wissen und Intelligenz sind verschiedene Dinge. Ich denke, das weiß er auch.}\\
``Und dann halt noch Chanahe-Sachen'', fügte er hinzu.\\
Inu blieb stehen. Der Wald um ihnen war seltsam still, aber das bemerkte das Mädchen nicht. Das 
Mondlicht, welches durch die Baumkronen fiel, zeichnete flimmernde Schatten auf den Waldboden. Ein 
kühler Wind ließ Inu sich fröstelnd über den Arm streichen. ``Du hast also deine Gabe gefunden?''\\
Semái schüttelte den Kopf und blickte sie dabei ganz ruhig an. ``Nein. Meine Mutter redet sich nur 
etwas ein. Sie meint, wenn ich lerne zu kämpfen, dann wird die Gabe schon kommen. Möglich, dass ich 
vielleicht ein bisschen mehr Geschick habe als die Lisoe, aber im Vergleich zu den Chanahe bin ich 
einfach nur mies.''\\
Inu wusste nicht recht, was sie darauf sagen sollte. Für einen Chanahe war es vermutlich das 
schlimmste was ihm passieren konnte, aber sie freute sich trotzdem ein bisschen, dass sie einander 
ähnlicher waren und nicht durch diese Begriffe getrennt werden konnten. \\
``Aber dann könnte deine Mutter dich doch einfach in Ruhe lassen. Dann könntest du wieder öfters 
raus.''\\
Semái schüttelte wieder den Kopf. ``Aber ich wollte mich bedanken. Ich weiß nicht, was passiert 
ist… ich kann mich nur noch dran erinnern, dass ich gefallen bin. Meroka hat gesagt, dass du dann 
noch da warst. Das du auch ohnmächtig geworden bist. Es tut mir Leid, meine Mutter hat nicht sehr 
freundlich von dir gesprochen, also kann ich mir vorstellen, was sie zu dir gesagt hat.''\\
Inu zuckte mit den Schultern und tat so, als ob die Worte und Handlungen der Frau sie nicht gestört 
hätten. Das konnte sie aber nicht lange verbergen und ihre Augen füllten sich mit Tränen. \\
``Ich… ich hatte einfach so Angst, dass du jetzt nie wieder was mit mir machen willst. Dass du mich 
nicht mehr magst…'', murmelte sie. \\
``Ich mag dich'', sagte Semái und drückte aufmunternd ihre Hand. ``Du bist ganz anders als die 
Chanahe. Die denken immer nur daran, dass sie besser sein wollen als die anderen Chanahefamilien. 
Aber wenn sie denen dann begegnen, sind sie ganz freundlich und lügen.''\\
``Aber warum?''\\
Der Junge zögerte kurz. ``Mein Vater sagt, es herrschte lange genug Anarchie. Er meint, dass die 
Chanahe sich zusammen schließen und für Ordnung sorgen sollen. Deshalb halten sie sich auf für 
besser als Lisoe. Sie meinen, die Lisoe müssen für sie arbeiten, weil sie nicht begnadet sind. 
Struktur ist wichtig, sagt er.''\\
Inu dachte einen Moment über diese Worte nach und schüttelte dann den Kopf. ``Aber wir waren doch 
schon immer frei. Warum soll das jetzt aufhören?''\\
``Sie meinen, dass es besser ist. Wenn jeder Mensch nur an seine Freiheit denkt, gibt es so viel 
Krieg und Tod. Wenn die Chanahe entscheiden, dann müssen sich die Lisoe nicht mehr so viel 
bekämpfen.''\\
``Ihr kämpft doch auch!''\\
``Ja… da planen sie gerade, dass zu ändern. Es sollen Arnankämpfe geben. Wie eine Sportart. Ach… 
ich  bekomm ja auch nicht alles mit. Aber Arashad ist eigentlich nur dafür da, für Ordnung zu 
sorgen. Vorher lebten wir ja auch nicht so viele auf einen Haufen.''\\
``Aber das ist nicht gut'', entschied Inu: ``Kanto schenkte manchen vielleicht ein paar blöde 
Gaben. Aber er schenkte allen Menschen die Freiheit!''\\
``Sie bringen sich gegenseitig um'', sagte Semái mit großen Augen: ``Ist es das wert?''\\
``Das ist doch ihr Problem. Sie müssen es ja nicht tun.''\\
Der Junge ließ die Schultern hängen. ``Ach… mir ist es langsam egal, was die Erwachsenen machen. 
Ich sollte dich nach Hause bringen. Sonst sucht dein Vater uns noch.''\\
``Das macht der nicht. Der ist dafür zu faul. Kommst du bald wieder? Tagsüber?''\\
Er grinste. ``Ich versuch‘s. Ich will dir die wilden Pferde zeigen! Das ist tagsüber besser.''\\
\textbf{Ich muss damit leben, dass ihr befreundet seid, hm?}, vermutete Saraé amüsiert.\\
\textit{Ja.}\\