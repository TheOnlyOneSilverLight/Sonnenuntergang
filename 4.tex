\chapter{Jenseits des Tals}

Das Feuer züngelte empor, leckte an dem feuchten Holz. Es schien, als wäre es verärgert, kein 
besseres Brennmaterial bekommen zu haben und so niedrig sein zu müssen. Dafür spuckte es umso mehr 
Qualm in die Luft. Semái stocherte gedankenverloren darin herum und lauschte dem stetigen Knacken, 
wenn die Luft aus dem Holz entwich. Sie hatten es von einer windschiefen Hütte geklaut, die 
keineswegs verlassen aussah. Irgendeine arme Jägersfamilie, die sich etwas abseits der Siedlung 
eine Unterkunft zusammen gezimmert hatte. Und, wie alle anderen Bewohner dieses Tals, sich bereits 
für das Herbstfest auf dem Dorfplatz eingefunden hatte.
Sie waren nun drei Tage unterwegs und folgten dem steinigen Weg hinauf zum Pass, der sie aus dem 
Tal bringen würde.\\
Semái lächelte grimmig und schüttelte kaum merklich den Kopf über den spielerischen Streit der 
beiden Frauen. Sie warfen sich sarkastische Beleidigungen an den Kopf und das Ziel war wohl, am 
originellsten zu sein. Eigentlich sah kein Gespräch zwischen den Beiden anders aus als dieses. Nur 
jetzt kicherte Revo an bestimmten Stellen amüsiert. Semái war sich nicht sicher wie ernst diese 
Geplänker waren. Manchmal machte er auch mit, aber heute hatte er keine Muse dazu. Der Chanahe legte 
den Kopf in den Nacken und versuchte den Weg des Qualms zu verfolgen. Er verschmolz mit der 
Nachtluft und dämpfte das Licht einiger Sterne. Semái fröstelte und zog die Schultern hoch.\\
\textit{Ein wärmeres Feuer wäre besser...}\\
``Warum bist du heute so still?'', fragte Inu: ``Also, noch stiller als sonst?''\\
Semái suchte nach den richtigen Worten, fand sie jedoch nicht. Er würde Inu kränken, wenn er ihre 
Frage ehrlich beantworten würde. Dies war die letzte Nacht, die sie in der Sicherheit des Tals 
verbringen würden. Hier, wo die schlimmsten Kreaturen zu groß geratene Feldratten oder wilde 
Raubkatzen waren. Er war damals noch ein Junge gewesen, aber er hatte die Kadaver der Wesen 
gesehen, die die Chanahe-Vorhut freigeräumt hatte. Er wusste noch genau, wie sich die Nächte 
angefühlt hatten. Die Kinder waren die Einzigen, die schliefen. Sein Vater war damals nicht einmal 
dazu bereit, sich auf dem Boden nieder zu lassen. Stetig lief er um da Feuer und starrte in die 
Nacht. Wohlwissend, dass in der Dunkelheit sie die Opfer waren und nicht die Jäger. Inu hatte 
keinen Schimmer. Ihre Welt existierte bisher nur bis zu dieser Bergkuppe. Er spürte Sarjus Blick 
auf sich ruhen. Eine stumme Aufforderung. Aber er antwortete bloß: ``Nichts. Ich habe nur 
gebetet.''\\
Inu schnaufte spöttisch und verschlang den letzten Bissen ihres Mahls. "Such dir lieber mächtigere 
Götter."\\
Semái musste sich ein Grinsen verkneifen und wandte den Blick ab um die verräterische Bewegung 
seiner Lippen vor ihr zu verbergen. Sarju machte sich nicht diese Mühe und lachte kurz laut auf, 
ehe sie sich weiter damit beschäftigte, Muster mit einem Stock in die Walderde zu ritzen.\\
Nur Revo beugte sich zu Inu vor und fragte nach. ``Was meinst du damit?''\\
``Kanto und Anta haben ihre drei Welten erschaffen. Aber es gibt noch viel mehr Götter, außerhalb 
und innerhalb'', erklärte Inu und warf Sarju einen bösen Blick zu: ``Götter, die mehr Macht 
besitzen.''\\
``Sie sind die Schöpfer. Warum sollten sie irgendetwas dulden, was mächtiger ist als sie?'', 
murmelte Semái.\\
``Wir interessieren sie nicht mehr'', entschied Inu und trat gegen einen herausragenden 
brennenden Scheit. Funkten stopbten auf und verglühten in der Luft.\\
``Das würden sie nicht tun'', erwiderte Semaí: ``Sie würden ihre Schöpfung nicht im Stich 
lassen. Sie sind besser als wir Menschen.''\\
``Aber sie haben dir die Gabe verwehrt'', warf Revo ein: ``Zeigt das nicht, dass sie dich für 
wertlos halten? Du bist auch nicht mehr als ein Lisoe!''\\
Semái runzelte die Stirn. ``Ich bezeichne Lisoe nicht als wertlos!''\\
``Aber in den Augen deiner Götter ist es doch so'', spottete Inu und nickte Revo zustimmend zu: 
``Sie haben zwei Arten von Menschen geschaffen. Und die Chanahe halten sich für besser. Begnadet. 
Dabei ist eure Macht so zerbrechlich. Ein Windhauch reicht und die Chanahe sind hilflose kleine 
Kinder.''\\
Semái sah sie überrascht an und versuchte die richtigen Worte für eine Antwort zu finden. Revo kam 
ihm zuvor: ``Ist das dein Ernst? Gibt es eine Möglichkeit, die Gaben zu zerstören?''\\
Bevor Inu etwas erwidern konnte, unterbrach Sarju das Gespräch mit einem lauten Stöhnen. 
``Ernsthaft? Hätte ich gewusst, dass ihr so drauf seid, hätte ich euch niemals angesprochen.''\\
``Sind wir auch nicht'', sagte Semái bestimmt und starrte Inu an. Er hatte sich dagegen 
entschieden, sie hier vor den Zwillingen zu fragen, was in sie geahren war. Natürlich hatte sie 
die Chanahe nie leiden können. Aber nie hatte sie Kanto und Anta verleumdet.\\
``Gute Nacht'', beschloss Sarju, ließ sich rücklings aus dem Sitz hinfallen und warf ihren Umhang 
über die Augen.\\
Inu biss sich mittlerweile unruhig auf die Unterlippe und schien ihre Worte zu bereuen. Ein 
schnelles, kurzes Handzeichen war die Antwort auf Semáis prüfenden Blick.\\
\textit{Können wir reden?}, bedeutete es und üblicherweise hatten sie es verwendet, um 
herauszufinden ob der andere Zeit hatte sich fortzuschleichen.\\
Semái wandte den Blick ab und legte sich, mit dem Rücken zum Feuer und somit zu Inu, ebenfalls zum 
schlafen hin. 

Inu hatte eine schlaflose Nacht hinter sich. Immer wieder spähte sie zu Semái hinüber und auch 
Saraés Stimme begleitete sie fortwährend. Um die einsamen Stunden zu nutzen, schürte sie das Feuer 
nach und sortierte ihre Ausrüstung ein weiteres Mal. Am Morgen bereitete sie sogar eifrig ein 
schnelles Frühstück für alle vor. Revo war der Einzige, der mit ihr sprach, bis seine Schwester ihn 
aufforderte, mit ihr auf eine letzte kurze Jagdrunde im Tal zu gehen. Die Chanahe nickte Inu zum 
Abschied nur zu. Eine stumme Aufforderung, das wieder ins Reine zu bekommen. Beklommen biss sich die 
junge Frau auf die Lippen. \textit{Du bist still!}, forderte sie Saraé auf und wandte sich an Semái, 
der neben Marai auf dem Boden saß und ganz versunken darin schien, sein Messer zu schärfen.\\
``Denkst du darüber nach, zu bleiben?'', fragte sie zögernd.\\
``Nein'', sagte er sofort und ohne aufzublicken: ``Die Fragen auf die ich eine Antwort suche sind 
eher: Wann hast du dich verändert und warum habe ich es nicht mitbekommen? Oder kannte ich dich 
nicht und sah in dir nur etwas, was ich sehen wollte?''\\
Seine Worte überforderten Inu. Ratlos stand sie vor ihm und drehte unruhig einen ihrer dünnen Zöpfe 
um die Finger. Saráe war es, die die passende Erwiderung fand und Inu wiederholte sie, ehe der 
Gedanke zu ende gesprochen war: ``Was hast du in mir gesehen?''\\
Semái musste wirklich lange über diese Fragen nachgedacht haben, denn anders als Inu erwartete, 
antwortete er ohne Zögern. ``Eine Träumerin. Mit dir konnte ich vor dem fliehen, was ich in meinem 
Leben nicht mochte. Wenn die Welt in Flammen stand, bist du hindurch geschwommen. Wenn die Erde 
bebte, hast du deine Flügel ausgebreitet und bist geflogen. Und du hast mich mitgenommen. Du hast 
mir erst gezeigt, was Glauben bedeutet. Und welche Macht es hat.''\\
``Nicht Kanto und Anta haben mir diese Macht gegeben'', murmelte Inu.\\
Semái stand auf und schulterte seinen Beutel. In seinem Blick lag etwas prüfendes. ``Dann erzähl 
es mir.''\\
In einer gleichzeitig hilflosen und wütenden Geste hob sie die Arme und ließ sie wieder fallen, 
trat einen Schritt energisch auf ihn zu, nur um wieder zu erstarren. ``Sie meint es gut mit uns.''\\
``Bist du dir sicher?''\\
``Saraé hat dir dein Leben gerettet'', entschied Inu und hob herausfordernd ihr Kinn: ``Damals, 
als du gestürzt bist. Wo war Kanto oder Anta?''\\
``Sie sind Schöpfer'', antwortete Semái: ``Keine Herrscher. Dieses Wesen ist bei dir?''\\
``Immer.''
Unschlüssig beobachtete sie ihn beim packen und wie er das Feuer austrat. ``Aber du begleitest 
mich?'', wiederholte sie.\\
Sein Blick richtete sich wieder auf Inu. ``Du hast keine 
Ahnung, was dich erwartet. Und deine Göttin scheint schwach zu sein, wenn sie keine eigene Gestalt 
hat.''\\
``Saraé hat dein Leben gerettet!'', unterbrach sie ihn.\\
``Kanto und Anta haben drei Welten erschaffen. Kanto und Anta haben Kristall und Glas belebt. Sie 
sind Sonne und Mond. Beantworte es dir selbst.''
``Warum gehst du dann nicht?''\\
Er hielt ihr das Messer entgegen. ``Wenn du leben willst, brauchst du mich und mein Messer. Sarju 
und ihre Schwerter. Keiner da draußen, außer Chanahe-Truppen, besitzen Klingen, Inu. Ihr Lisoe 
verflucht die Clans, weil sie euch eure Freiheit nehmen. Ohne Ordnung seid ihr nicht mehr als ein 
Haufen Gesetzloser. Ihr habt weder das Wissen noch die Fähigkeit euch weiter zu entwickeln, denn ihr 
seid zu sehr damit beschäftigt euch zu prügeln. Ich bin noch nicht bereit dazu, die Inu, die ich 
dachte zu kennen, dem Chaos allein begegnen zu lassen.''\\
``Ich bin noch die Selbe'', platze es aus Inu heraus.\\
Semái nickte. ``Gut.''

Sie folgten den Pfaden der Tiere. Der Wald war zu dicht um weit blicken zu können, aber stetig 
führte der Weg bergauf. Sarju verschwand immer mal wieder, sprang über Geröll und kletterte an 
Bäumen hinauf um sich umzusehen. Inu bemerkte das Strahlen in ihren Augen, wenn sie von ihren 
Aussichtspunkten zurück kehrte und wurde immer grimmiger. Was für ein Gefühl es wohl sein musste, 
fliegen zu können...\\
``Wir erreichen bald den Pass'', erklärte Sarju: ``Wir sollten uns nur eine kurze Pause erlauben. 
Sonst schaffen wir die Überquerung heute nicht mehr.''\\
``Haben wir es eilig?'', fragte Inu: ``Das Ende der Welt wird nicht vor uns weglaufen.''\\
Sarju wollte etwas erwidern, aber Semái war schneller: ``Nein, aber wir wählen einen Pass außerhalb 
der Chanahe Patrouillen. Sie sind gefährlicher und ohne Markierungen. Selbst bei Tageslicht müssen 
wir vorsichtig genug sein.''\\
\textit{Er meint, alle außer Sarju,} dachte Inu grimmig, aber nickte nur.\\
Die Baumgrenze kam überraschend plötzlich. Fasziniert blieb Inu stehen und betrachtete die 
Blätterkronen unter ihr. Sie meinte sogar, Häuser im Tal erkennen zu können. War das da am Horizont 
die Stadt?\\
\textit{Wir sind so klein,} wurde ihr bei diesem Ausblick bewusst.\\
\textbf{Ich mache dich zu etwas Großem}, wisperte Saraé.\\
Unwillkürlich nickte Inu. Eine warme Hand berührte ihren Ellbogen und sie wandte sich um. Fragend 
blickte Semái sie an. Inu sah über seine Schulter und erkannte die Zwillinge hinter einem Felsen 
verschwinden. Dort musste es zum Pass führen.\\
``Du kannst deine Meinung noch ändern'', sagte er leise.\\
Sie schüttelte den Kopf. ``Nein, ich... habe nur überlegt, wie es wohl sein muss, zu fliegen.''\\
Semái zeigte sein schiefes Grinsen. ``Erst das Ende der Welt, dann fliegen lernen, ja?''\\
Inu lächelte flüchtig und hoffte, dass dieses Gespräch bedeutete, dass ihr Streit beigelegt war. 
Schnell ergriff sie seine Hand und zog ihn weiter das Geröll hinauf. Auf den Felsen zu, hinter dem 
Revo und Sarju verschwunden waren. Es offenbarte sich ein schmaler Pass, umgeben von hoch 
aufragenden Felswänden. Der Weg war nicht mehr als loser Kies, unterbrochen von größeren Steinen und 
Felsen. Weiterhin ging es Bergauf. Ihre Schritte ließ Geröll sich lösen und unter ihnen hinweg 
bröckeln.\\
``Es ist kalt'', murmelte Inu und rieb sich die Hände. Ihr Atem wurde zu einer weißen Wolke und 
stieg in den Himmel empor. Sogar einzelne Flecken Schnee und Frost bedeckten die ein oder andere 
Felsformation.\\
``Und was soll hier gefährlich sein?'', redete sie weiter, nachdem Semái nicht reagierte. Er war zu 
sehr damit beschäftigt, sich umzusehen. ``Außer zu stolpern.''\\
Nach einem flüchtigen Blick in den Himmel erwiderte er trocken: ``Harpyien.''\\
Inu schwieg. Sie hatte diesen Wort noch nie gehört, wollte es ihm gegenüber aber nicht zugeben. 
Aber von nun an schwieg sie und ihr Blick wanderte ebenfalls immer wieder über den Himmel. Es 
dauerte nicht lang, da führte der Weg wieder bergab. Drei mal knickte Inus Knöchel um. Verbissen 
presste sie die Lippen aufeinander und ignorierte das Ziehen in den Gelenken. Rasch wurde der Pfad 
steiler, aber dafür verschwand auch das Geröll. Sie vermutete, dass hier im Frühling ein Bach aus 
Schmelzwasser entlang floss und alles lose Gestein mit sich nahm.\\
Ein Felsvorsprung ragte wie der Zahn einer Schlange hervor. Als sie darum herum gingen, trafen sie 
wieder auf die Zwillinge. Sarju kniete am Boden, während Revo unruhig daneben stand. Semái 
reagierte ganz anders. Er trat einen Schritt zurück an die Felswand und suchte hektisch den Himmel 
ab, bis Sarju sagte: ``Ist schon ne Weile her. Mindestens drei Tage. Aber wir sollten wirklich 
nicht riskieren, hier die Nacht zu verbringen.''\\
Inu schob sich an Revo vorbei um endlich sehen zu können, worüber die Chanahe sprach. Sie bereute 
es gleich, konnte aber den Blick nicht von der Leiche - oder waren es zwei? - lösen. Fetzen von 
trockenem, rötlichen Fleisch hingen an Gliedmaßen, deren blanker Knochen man sehen konnte. Als 
hätten Schnäbel in den Körper gehackt. Hätte man nicht noch einen Ärmel erkannt, hätte 
Inu sich eingeredet, dass es ein Tier wäre.\\
``Da wollte wohl jemand ins Tal'', erklärte Revo als Antwort auf Inus irritiertem Blick.\\
Inu beschlich ein Verdacht. ``Warum patrouillieren die Chanahe auf den Pässen? Wegen diesen 
Ungeheuern?''\\
``Ja'', erwiderte Semái knapp, warf einen letzten Blick zu den Wolken und eilte dem Weg entlang 
weiter. Es war Sarju, die ihre Frage beantwortete. ``Ihr habt da ein kleines Wunder, Inu. Euer 
Tal... so abgeschottet von Ungeheuern. So leicht zu verteidigen. Eine Basis, um Zivilisation zu 
erschaffen. Aber es gibt nicht unendlich viel Platz. Die Ressourcen sind nicht unerschöpflich.''\\
Sie biss sich auf die Zunge und starrte auf die zerfetzte Leiche. Ihre Hände ballten sich zu 
Fäusten. \textit{Ein Lisoe. Natürlich.}\\
``Ihr könnt uns nicht ausrotten'', fauchte sie schließlich: ``Ihr werdet untergehen! Eure Zeit ist 
längst vorbei!''\\
``Hey'', mahnte Revo und trat zwischen sie und seiner Schwester. Beschwichtigend hob er die Arme 
und sah sie eindringlich an. ``Ich teile deinen Zorn. Und Sarju auch, sonst wär sie nicht hier.''\\
\textit{Du bist dumm}, fällte Inu ihr Urteil über den gabenlosen Chanahe. Auch wenn Revo den Blick 
seiner Schwester nicht bemerkte, Inu tat es. Und beide Frauen wussten ganz genau, warum die 
Begnadete sich zur Flucht entschieden hatten. Wäre ihr Bruder nicht ein Verbannter, hätte sie 
bereitwillig da Spiel der Chanahe in ihrem kleinen, sicheren Tal fortgeführt.\\
``Es wird spät'', brach die Begnadete die Stille, ordnete ihre Schwerter neu und kehrte den Leichen 
den Rücken zu.



